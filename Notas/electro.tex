\part{Electromagnetismo}

\vspace*{\fill}

\begin{center}
	\textit{''El conocimiento es una red de ideas interconectadas, una vez que atrapamos una, las demás vienen detrás'' - Michael Faraday.}
\end{center}

\vspace*{\fill}

\chapter{Análisis Vectorial}
\dsnote{Solo se enunciarán los teoremas y propiedades más importantes.}
\section{Operadores Vectoriales}
\subsection{Gradiente}
El gradiente de un campo escalar es un campo vectorial. Es un vector normal a la curva de nivel en el punto estudiado
\begin{equation}
	\grad{f(r)} = \qty(\pdv{f}{x_1} \vu{x}_1 + \cdots + \pdv{f}{x_n} \vu{x}_n),
\end{equation}
Cuando $\grad{f} = 0$ se tiene un punto estacionario (máximo o mínimo). Dicho de otra forma, el gradiente marca la dirección en la cual varía más rápido el campo escalar y su magnitud indica cuan rápido varía. \\
\subsubsection{En Diferentes Sistemas de Coordenadas}
\begin{align*}
	\grad{\varphi} &= \pdv{\varphi}{r} \vu{r} + \frac{1}{r} \pdv{\varphi}{\theta} \vu{\theta} + \frac{1}{r\sin{\theta}} \pdv{\varphi}{\phi} \vu{\phi} \qquad \text{Coordenadas Esféricas} \\
	\grad{\varphi} &= \pdv{\varphi}{\rho} \vu{\rho} + \frac{1}{\rho} \pdv{\varphi}{\phi} \vu{\phi} + \pdv{\varphi}{z} \vu{z}.
\end{align*}
 


\subsection{Divergencia}
La divergencia de un campo vectorial mide la diferencia entre el flujo saliente y entrante en una región dada.
\begin{equation}
	\div{A} = \pdv{A_x}{x} + \pdv{A_y}{y} + \pdv{A_z}{z}.
\end{equation}
La divergencia produce un escalar, mide cuanto un vector se dispersa o sale de un punto.
\subsubsection{En Diferentes Sistemas de Coordenadas}
\begin{align*}
	\div{A} &= \frac{1}{r^2} \pdv{(r^2 A_r)}{r} + \frac{1}{r\sin{\theta}} \pdv{(\sin{\theta} A_\theta)}{\theta} + \frac{1}{r\sin{\theta}} \pdv{A_\phi}{\phi} \qquad \text{Coordenadas Esféricas} \\
	\div{A} &= \frac{1}{\rho} \pdv{(\rho A_\rho)}{\rho} + \frac{1}{\rho} \pdv{A_\phi}{\phi} + \pdv{A_z}{z} \qquad \text{Coordenadas Cilíndricas.}
\end{align*}








\subsection{Rotacional}
Operador vectorial sobre campos vectoriales: tendencia de un campo vectorial a inducir rotación.
\begin{equation}
	\curl{A} = \mqty| \vi & \vj & \vk \\ \pdv{x} & \pdv{y} & \pdv{z} \\ A_x & A_y & A_z |
\end{equation}


\section{Integración Vectorial}
Integrales de línea, superficie y volumen. Integrandos pueden ser vectores o escalares. \\
\subsection{Integrales de Línea}
Recorridos o trayectorias
\begin{equation}
	\int _c F(r) \cdot \dd{l}.
\end{equation}
Donde $F$ es un campo vectorial, $\dd{l}$ desplazamiento vectorial infinitesimal a lo largo de la curva y $c$ curva sobre la cual se integra. Cuando la integral de línea es independiente de la trayectoria el campo es conservativo.

\subsection{Integrales de Superficie}
Esta integral mide un flujo
\begin{equation}
	\int _S F\cdot \vu{n} \dd{a}
\end{equation}
Esto nos da un escalar. Si laintegral es cerrada $\oint F\cdot \vu{n} \dd{a}$.

\subsection{Teoremas Importantes}
\begin{teorema}
	\textbf{Teorema Fundamental de la Divergencia: }
	\begin{equation}
		\int _V \div{F} \dd{V} = \oint _S F\cdot \vu{n} \dd{a}.
	\end{equation}
\end{teorema}


\begin{teorema}
	\textbf{Teorema de Stokes: }
	\begin{equation}
		\int _S (\curl{F}) \cdot \vu{n} \dd{a} = \oint _c F \cdot \dd{l}.
	\end{equation}
\end{teorema}




\chapter{Electrostática}
Esta estudia los efectos mutuos que se producen entre los cuerpos como consecuencia de us carga eléctrica. La carga eléctrica es una propiedad fundamental y característica delas partículas elementales.
\begin{enumerate}
	\item La carga no se crea ni se destruye.
	\item Hay dos clases: positiva y negativa.
	\item En un sistema cerrado la carga se conserva.
\end{enumerate}

\section{Ley de Coulomb}
Experimentalemnte
\begin{enumerate}
	\item Dos cargas puntuales ejercen fuerza a lo largo de la línea que las une e inversamente proporcional al cuadrado de la distancia.
	\item La fuerza es proporcional al producto de las cargas.
\end{enumerate}

\begin{equation}
	\vec{F} = \frac{1}{4\pi \varepsilon _o} \frac{q_1 q_2}{r_{12} ^2} \vu{r}_{12}.
\end{equation}
Esto es válido para cargas puntuales, pero para muchas cargas se aplica el principio de superposición. Ahora, si la carga se distribuye en un volumen $V$ con densidad $\rho$ y sobre la superficie $S$ que limita $V$ con densidad $\sigma$, la fuerza ejercida por esta distribución de carga sobre una carga puntual $q$ está dada por:
\begin{align}
	F(r) &= \frac{q}{4\pi \ep} \int _V \rho (r') \frac{r - r'}{\abs{r - r'}^3} \dd{V'} + \frac{q}{4\pi \ep} \int _S \sigma (r') \frac{r - r'}{\abs{r - r'}^3} \dd{S'}.
\end{align}


\section{Campo Eléctrico}
Campo vectorial físico $\to$ región del espacio en la que interactúa la fuerza eléctrica. Se genera por cargas o campos magnéticos variables en el tiempo.
\begin{equation}
	\vec{E} (r) = \frac{1}{4\pi \ep} \sum _{i=1} ^N \frac{q_i}{r_i ^2} \vu{r} _i.
\end{equation}
Entonces, de manera general, tomando en cuenta distribuciones continuas de carga y cargas puntuales
\begin{equation}
	\vec{E} (r) = \kel \sum _{i=1} ^N q_i \frac{r - r_i}{\abs{r - r_i}^3} + \kel \int _V \frac{r - r'}{\abs{r - r'}^3} \rho (r') \dd{V'} + \kel \int _S \frac{r - r'}{\abs{r - r'}^3} \sigma (r') \dd{a'}
\end{equation}



\section{Potencial Electrostático}
El campo eléctrico cumple con
\begin{equation}
	\curl{\vec{E}} = 0.
\end{equation}
Sabemos que $\vec{E} (r) = -\grad{\varphi}$ es un potencial electrostático. Pero con esto se tiene
\begin{equation}
	\varphi (r) = -\int _{\mathcal{O}} ^r \vec{E} \cdot \dd{\vec{l}}
\end{equation}
donde $\mathcal{O}$ es el punto de referencia. Notemos que

\begin{enumerate}
	\item Potencial electrostático no es lo mismo que energía potencial.
	\item $\vec{E}$ es un vector derivado de un escalar y ya que $\curl{E} = 0$ esto implica: las componentes de $\vec{E}$ no son independientes
	\begin{equation}
		\pdv{E_x}{y} = \pdv{E_y}{x}, \qquad \pdv{E_z}{y} = \pdv{E_y}{z}, \qquad \pdv{E_x}{z} = \pdv{E_z}{x}.
	\end{equation}
	\item EL sistema de referencia es fundamental.
	\item El potencial obedece el principio de superposición.
	\begin{equation}
		\varphi (r) = \kel \sum_{i=1} ^N \frac{q_i}{\abs{r - r_i}} + \kel \int _V \frac{\rho (r')}{\abs{r-r'}} \dv{V'} + \kel \int _S \frac{\sigma (r')}{\abs{r - r'}} \dd{a'}.
	\end{equation}
\end{enumerate}

\section{Ley de Gauss}
Importante relación entre la integral de la componente normal del campo eléctrico sobre una superficie cerrada y la carga total encerrada por la superficie.
\begin{equation}
	\Psi _E = \int _S \vec{E} \cdot \vu{n} \dd{a}, \qquad \text{Flujo Eléctrica}
\end{equation}
Para una superficie cerrada imaginaria (superficie gaussiana). Se tiene que $\Psi _E \propto q$. Entonces
\begin{equation}
	\oint \vec{E} \cdot \vu{n} \dd{a} = \frac{1}{\ep} Q_{enc}.
\end{equation}
y por el teorema de la divergencia
\begin{equation}
	\div{\vec{E}} = \frac{\rho}{\ep}.
\end{equation}
A notar:
\begin{enumerate}
	\item Simetría esférica: Superficie gausiana $\to$ esfera concéntrica.
	\item Simetría cilíndrica: Superficie gaussiana $\to$ cilindro coaxial.
	\item Simetría palna: Superficie gaussiana $\to$ caja.
\end{enumerate}



\section{Dipolo Eléctrico}
Dos cargas iguales de signo contrario separados por una pequeña distancia. Supongamos una carga $-q$ en $r'$ y una carga $q$ en $r' + l$. Entonces
\begin{equation}
	\vec{E} (r) = \kel q \qty(\frac{r - r' - l}{\abs{r - r' - l}^3} - \frac{r - r'}{\abs{r - r'}^3})
\end{equation}
expandiendo el término de dentro
\begin{equation}
	\vec{E} (r) = \kel q \qty[\frac{3 (r - r') l}{\abs{r - r'}^3} (r - r') - \frac{l}{\abs{r - r'}} + \cdots]
\end{equation}

\subsection{Momento Dipolar Eléctrico}
Si se coloca un dipolo en un campo $\vec{E}$, ambas cargas $(q,-q)$ separadas una distancia $l$, experimentan fuerzas de igual magnitud y dirección contraria $\vec{F}$ y $-\vec{F}$, por ende: $\sum vec{F} = 0$ y $\sum \tau \neq 0$. Definimos el \textbf{momento dipolar} como $\vec{p} = q\vec{l}$. Por ende
\begin{align}
	\vec{E} (r) &= \kel \qty[\frac{3(r - r')\cdot \vec{p}}{\abs{r - r'}^3} (r - r') - \frac{\vec{p}}{\abs{r - r'}^3}].	\\
	\varphi (r) &= \kel q \qty[\frac{1}{\abs{r - r' - l}} - \frac{1}{\abs{r - r'}}] = \kel \frac{(r - r') \cdot \vec{p}}{\abs{r - r'}^3}.
\end{align}


\section{Trabajo y Energía en Electrostática}
Suponga una configuración de cargas estacionarias y se desea mover una carga de prueba $Q$ de un punto $a$ a un punto $b$. Por ende
\begin{equation}
	\varphi (b) - \varphi (a) = \frac{W}{Q}
\end{equation}
La diferencia de potencial entre $a$ y $b$ es igual al trabajo por unidad de carga requerido para mover $Q$ de $a\to b$.

\subsection{Energía de una Distribución de Cargas Puntuales}
Energía de la distribución $\to$ trabajo para ensamblar la distribución.
\begin{equation}
	W = \frac{1}{2} \sum_{i=1} ^n q_i \varphi (r_i).
\end{equation}
de otra forma se puede reescribir el trabajo
\begin{align}
	W &= \frac{\ep}{2} \qty[\oint _S \varphi \vec{E} \cdot \dd{a} + \int _V E^2 \dv{V}] \\
	W &= \frac{\ep}{2} \int _V E^2 \dv{V} \qquad \text{Si el volumen crece } (r\to \infty).
\end{align}





\chapter{Problemas Electrostáticos}

Estos problemas no son secillos de resolver. Por otro lado:
\begin{equation}
	\laplacian{\varphi} = -\frac{\rho}{\ep} \qquad \text{Ecuación de Poisson}.
\end{equation}

\textbf{Laplaciano:}
\begin{enumerate}
	\item \textbf{Rectangulares:}
	\begin{equation}
		\laplacian{\varphi} = \pdv[2]{\varphi}{x} + \pdv[2]{\varphi}{y} + \pdv[2]{\varphi}{z}.
	\end{equation}
	\item \textbf{Esféricas: }
	\begin{equation}
		\laplacian{\varphi} = \frac{1}{r^2} \pdv{r} \qty(r^2 \pdv{\varphi}{r}) + \frac{1}{r^2 \sin{\theta}} \pdv{\theta} \qty(\sin{\theta} \pdv{\varphi}{\theta}) + \frac{1}{r^2 \sin{\theta}} \pdv[2]{\varphi}{\phi}.
	\end{equation}
	\item \textbf{Cilíndricas: }
	\begin{equation}
		\laplacian{\varphi} = \frac{1}{\rho} \pdv{\rho} \qty(\rho \pdv{\varphi}{\rho}) + \frac{1}{\rho ^2} \pdv[2]{\varphi}{\theta} + \pdv[2]{\varphi}{z}.
	\end{equation}
\end{enumerate}

Y cuando interesa conocer el potencial en regones con $\rho = 0$.
\begin{equation}
	\laplacian{\varphi} = 0 \qquad \text{Ecuación de Laplace.}
\end{equation}

\section{Ecuación de Laplace}
\begin{teorema}
	Si $\varphi _1 ,\ldots ,\varphi _n$ son todas soluciones de la ecuación de Laplace, entonces:
	\begin{equation}
		\varphi = c_1 \varphi _1 + \cdots + c_n \varphi _n,
	\end{equation}
	con $c_i = $ ctes, también es solución de la ecuación de Laplace
	\begin{equation}
		\laplacian{\varphi} = c_1 \laplacian{\varphi}_1 + \cdots + c_n \laplacian{\varphi}_n = 0.
	\end{equation}
\end{teorema}


\begin{teorema}
	Dos sluciones de la ecuación de Laplace que satisfacen las mismas condiciones de frontera difieran a lo sumo en una constante aditiva.
\end{teorema}


\subsection{Ecuación de Laplace en una Dimensión}
Si $\varphi$ es una función de una variable. Las soluciones para cada uno de los sistemas importantes de coordenadas

\begin{enumerate}
	\item \textbf{Rectangulares:}
	\begin{equation}
		\varphi (x) = ax + b.
	\end{equation}
	\item \textbf{Esféricas: }
	\begin{equation}
		\varphi (r) = -\frac{a}{r} + b.
	\end{equation}
	\item \textbf{Cilíndricas: }
	\begin{equation}
		\varphi = a\ln{\abs{\rho}} + b.
	\end{equation}
\end{enumerate}

\subsubsection{Caso Esférico}
Para el caso esférico en dos dimensiones su solución luego de la separación de variables es
\begin{equation}
	\varphi (r,\theta) = \sum _{n=0} ^\infty \qty(A_n r^n + \frac{B_n}{e^{n+1}}) P_n (\cos{\theta}).
\end{equation}

\subsubsection{Caso Cilíndrico}
Ahora para el caso cilíndrico en dos dimensiones, su solución es:
\begin{equation}
	\varphi (\rho ,\theta) = A_o + B_o \ln{\abs{\rho}} + \sum _{n=1} ^\infty \qty(A_n \rho ^n + B_n \rho ^{-n}) \qty(C_n \cos{n \theta} + D_n \sin{n\theta}).
\end{equation}

\subsubsection{Caso Cartesiano}
Ahora la solución del caso cartesiano
\begin{equation}
	\varphi (x,y,z) = A e^{-(k + m) ^{1/2} x} \cos{m^{1/2} y} \cos{k^{1/2} z}.
\end{equation}

\section{Imágenes Electrostáticas}
Para un conujunto dado de condicioens de frontera, la solución a la ecuación de Laplace es única y resolviendo para $\varphi$ se ha encontrado la solución completa al problema. \\
El método de imágenes es un procedimiento para lograr este resultado sin resolver específicamente la ecuacioń diferencial. \dsnote{No se aplica universalmente, solo un número considerable de problemas (que bueno xd).} \\

Supongamos una carga $q$ arriba de un plano conductor: La carga $q$ inducirá una carga en el conductor. La potencia total a una distancia $\vec{r}$ estará dado por la contribución de la carga más la inducida en el conductor:
\begin{align*}
	\varphi (r) &= \varphi _1 (r) + \kel \int _S \frac{\sigma (r')}{\abs{r - r'}} \dd{a'}. \\
	\varphi (r) &= \varphi _1 (r) + \varphi _2 (r).
\end{align*}
$\varphi _2 (r)$ puede ser sustituído por un potencial debido a una distribución de carga especificada: 
\begin{itemize}
	\item Carga Imagen.
	\item Debe cumplir con las condiciones de frontera.
\end{itemize}


\subsection{Sistemas Conductores y Coeficientes de Potencial}
Cuando se tienen conductores con formas complicadas las soluciones analíticas quedan descartadas: Métodos numéricos. Se pueden sacar algunas conclusiones, supongamos $N$ conductores en una geometría fija:
\begin{itemize}
	\item Existe una relación lineal entre el potencial de un conductor y las cargas de los diversos conductores del sistema. $N$ conductores descargados excepto el conductor $j\to$ carga $Q_{jo}$. La solución de Laplace es el espacio fuera de los conductores la expresamos como:
	\begin{equation}
		\varphi _1 ^{(j)} (x,y,z) \quad \to \quad \text{potencial generado por } Q_{jo}.
	\end{equation}
	El potencial de cada uno de los conductores estará dado por:
	\begin{equation}
		\varphi _1 ^{(j)}, ldots, \varphi _N ^{(j)}
	\end{equation}
	Reexpresando la carga del $j-$ésimo
	$$ \lambda Q_{jo}, \quad \lambda = \text{cte}, \quad \text{sadisface la ecuación de Laplace.} $$
	\item El potencial también se multiplica por $\lambda $.
	\item Todas las derivadas se multiplica por $\lambda$.
	\item $\sigma 0 -\ep \pdv{\varphi}{\vu{n}} \to$ todas las densidades se multiplican por $\lambda$.
	\item Potencial de cada conductor es proporcional a $Q_j$
	\begin{equation}
		\varphi _i ^j = p_{ij} Q_j, \qquad (i = 1,2,\ldots ,M)
	\end{equation}
	donde $p_{ij}$ es constante que depende de la geometría utilizando el mismo argumento para conductor $k$ $Q_k = \nu Q_{k_o}$ con $\nu =$ cte. Generalizando
	\begin{equation}
		\varphi _i = \sum _{j=1} ^{M} p_{ij} Q_j.
	\end{equation}
\end{itemize}









\chapter{Campo Electrostático en Medios Dieléctricos y su Teoría Microscópica}
Existen materiales conductoresy aislantes (dieléctricos)
\begin{itemize}
	\item Material dieléctrico ideal: material que no tiene cargas libres.
	\item Bajo la presencia de un campo eléctrico externo: pueden tener pequeños desplazamientos.
	\item Dieléctrico se polariza. Un dieléctrico polarizado es eléctricamente neutro, sin embargo produce un campo eléctrico en los puntos exteriores en interiores del dieléctrico. En un caso extermo, bajo un campo muy grande, ocurre la ionización. Un dipolo inducido $\vec{p} = \alpha \vec{E}$ donde $\alpha$ es la polarizabilidad atómica, depende de la estructura del átomo. Para moléculas es más complejo debido a que tienen direcciones preferenciales $\alpha \to \alpha _{ij}$.
\end{itemize}

\section{Polarización}
Consideremos un pequeño volumen $\Delta V$ de un medio dieléctrico que como todo es eléctricamente neutro. Si el medio está polarizado: Momento dipolar eléctrico
	\begin{equation}
		\Delta \vec{p} = \int _{\Delta V} r\dd{q}.
	\end{equation}
$\Delta \vec{p}$ depende del tamao del elemento de volumen. Se introduce el momento dipolar eléctrico por unidad de volumen y se conoce como polarización eléctrica:
\begin{equation}
	P \equiv \frac{\Delta \vec{p}}{\Delta V}.
\end{equation}
Momento dipolar de una molécula
\begin{equation}
	\vec{p} _m = \int _{\text{milecula}} r\dd{q}.
\end{equation}



\subsection{Campo Fuera de un Medio Dieléctrico}
Consideramos una porción finita de material dieléctrico polarizado. Esta polarización genera un campo eléctrico $\vec{E} \to \varphi$. Cada elemento de $\Delta V'$ se caracteriza por un momento dipolar $\Delta \vec{p} = P\Delta V'$ y tenemos $r \gg r'$. La contribución de las cargas en $\Delta V'$ al potencial está dada por:
\begin{equation}
	\Delta \varphi (r) = \kel \frac{\Delta p \cdot (r - r')}{\abs{r - r'}^3} = \kel \frac{P(r') \cdot (r - r') \Delta V'}{\abs{r - r'}^3}.
\end{equation}
El potencial total es la suma de todas las contribuciones de todas las partes del dieléctrico:
\begin{equation}
	\varphi (r) = \kel \int _{V_o} \frac{P(r') (r - r')}{\abs{r - r'}^3} \dd{V'}.
\end{equation}
Ahora buscamos encontrar $\vec{E}$
\begin{equation}
	\frac{r - r'}{\abs{r - r'}^3} = \nabla ^\prime \qty(\frac{1}{\abs{r - r'}}).
\end{equation}
Desarrollando la siguiente identidad: $\nabla ^\prime (f\vec{F}) = f\nabla ^\prime \cdot \vec{F} + \vec{F} \cdot \nabla ^\prime f.$ Se desarrolla la integral anterior, con ello se llega a 
\begin{equation}
	\varphi (r) = \kel \oint _{S_o} \frac{P \cdot \vu{n} \dd{a}}{\abs{r - r'}} + \kel \int _{V_o} \frac{(-\nabla ^\prime \cdot P) \dd{V}}{\abs{r - r'}}
\end{equation}
Con lo que se definen dos nuevas funciones escalares
\begin{itemize}
	\item $\sigma _p = P \cdot \vu{n}$.
	\item $\rho _p = -\div{P}$.
\end{itemize}

\begin{itemize}
	\item La densidad superficial de carga de polarización está dada por la componente de polarización normal a la superficie.
	\item La densidad de carga de polarización volumétrica es una medida de la no uniformidad de la polarización dentro del material.
\end{itemize}

Luego, calculando el campo eléctrico
\begin{equation}
	\vec{E} = \kel \qty[\int _S \sigma _p \frac{r - r'}{\abs{r - r'}^3} \dd{a'} + \int _{V_o} \rho _p \frac{r - r'}{\abs{r - r'}^3} \dd{V'}].
\end{equation}


\subsection{Campo Eléctrico dentro de un Dieléctrico}
En un dieléctrico la carga de prueba es comparable al tamaño de las moléculas. EL campo eléctrico dentro del dieléctrico debe tener las mismas propiedades. El campo eléctrico en un dieléctrico es igual al campo eléctrico dentro de una cavidad. \\

La ley de Gauss en un dieléctrico viene dada por
\begin{equation}
	\oint (\ep \vec{E} + \vec{P}) \cdot \vu{n} \dd{a} = Q.
\end{equation}
donde el término $\vec{D} = \ep \vec{E} + \vec{P}$ se le denomina \textbf{desplazamiento eléctrico}. 


\subsection{Suceptibilidad Dieléctrica y Constante Dieléctrica}
La polarización de un medio dieléctrico ocurre en respuesta al campo eléctrico en el medio. El grado de polarización depende:
\begin{itemize}
	\item Campo eléctrico
	\item Propiedades del material
\end{itemize}
A nivel macroscópico $F = P(E)$. En la mayoría de materiales $P$ se anula cuando $\vec{E} = 0$. Para materiales de este tipo y si son materiales siótropos, la polarización tendrá el mismo sentido que $\vec{E}$. 
\begin{equation}
	P = \chi (E) \vec{E} = \ep \chi _e \vec{E}.
\end{equation}
con esto se define $\varepsilon = \ep (1 - \chi _e)$.


\section{Condiciones de Frontera para los Vectores de Campo}
Variación de $\vec{E}$ y $\vec{D}$ al pasar por una zona interfacial entre dos medios. Considerando dos emdios encontacto y una densidad superficial de carga externa $\sigma$. Construir una pequeña superficie $S$: forma de caja de pastillas de altura despreciable. Entonces
\begin{equation}
	D_{2n} - D_{1n} = \sigma
\end{equation}
Observaciones
\begin{itemize}
	\item La discontinuidad en la componente normal $\vec{D}$ está dada por la densidad superficial de carga en la zona interfacial.
	\item Si no hay carga en la zona interfacial la componente normal de $\vec{D}$ no es contínua.
\end{itemize}
Y por la ley de Gauss se concluye que la componente tangencial del campo eléctrico es continua.
\dsnote{Después de toda esta parafernalia, el ejemplo clásico es el de la esfera dieléctrica en un campo eléctrico extermo, revisar el libro.}



\section{Teoría Microscópica de los Dieléctricos}
\subsection{Campo Molecular en un Dieléctrico: $E_m$}
Es el campo eléctrico en una posición molecular del dieléctrico el cual es producido por todas las fuentes externas y por todas las moléculas polarizadas del dieléctrico con excepción de la molécula en el punto considerado. El dieléctrico se polariza al inducir un campo. Suponemos polarización uniforma $\div{\vec{P}} = 0$. El campo eléctrico en el centro de la cavidad puede expresarse como:
\begin{equation}
	\vec{E}_m = \vec{E} _x + \vec{E}_d + \vec{E} _s + \vec{E} ^\prime
\end{equation}

\begin{itemize}
	\item $E_x$ campo eléctrico primario debido a los planos.
	\item $E_d$ campo debido a la carga de polarización e la superficie.
	\item $E_s$ campo debido a la carga de polarización en la superficie $S$.
	\item $E^\prime$ campo generado por dipolos dentro de $S$.
\end{itemize}


\subsection{Moléculas Polares}
\begin{itemize}
	\item Momento dipolar permanente
	\item Están formadas por al menos dos especcies distintas de átomos.
	\item En ausencia de campo eléctrico por una porción macroscópica del dieléctrico polar no está polarizada: dipolos individuales orientados al azar.
\end{itemize}
Si el dieléctrico polar se somete a un campo eléctrico, los dipolos se alinean con el campo. Si el campo es lo suficientemente intenso, la polarización alcanza el vapor de saturación: 
\begin{equation}
	P_s = N\vec{p}_m
\end{equation}
Se requiere valores de campo muy intensos. Si la temperatura se eleva la polarización disminuye. Según la mecánica estadística, a una temperatura $T$, la probabilidad de encontrar una molécula con energía $E$ es:
\begin{equation}
	f(E) = \propto e^{-E/kT}
\end{equation}
La energía potencial de un dipolo permanente $p_o$ en un campo eléctrico es:
\begin{equation}
	u = -p_o \cdot E_m
\end{equation}
La energía cinética de las moléculas no dependen del campo, entonces se desprecia su contribución en la distribución. El moemtno diplar efectivo de un dipolo molecular es su componente en la dirección del campo: $p_o \cos{\theta}$. El valor promedio de la cantidad está dado por:
\begin{equation}
	\expval{x}  =\sum x_m p_m = \frac{\sum _m \chi _m e^{-\beta e_m}}{\sum _m e^{-\beta u_m}}.
\end{equation}

pasando a lo continuo
\begin{equation}
	\expval{p_o \cos{\theta}} = p_o \qty(-\frac{1}{y} + \coth{y})
\end{equation}
con $y = p_o \frac{E_m}{kT}$ y a esta fórmula se le conoce como \textbf{Fórmula de Lagevin}. y el momento dipolar efectivo promedio
\begin{equation}
	\expval{p_o \cos{\theta}} = \frac{p_o ^2 E_m}{3kT}.
\end{equation}
Polarizabilidad por orientación $\alpha = \frac{p_o ^2}{3kT}$.

\begin{equation}
	\alpha = \alpha _o + \frac{p_o ^2}{3kT}
\end{equation}
a esta se le conoce como la \textbf{ecuación de Langevin-Debye}.

\subsection{Polarización Permanente: Ferroelectricidad}
Sabemos que
\begin{equation}
	E_m = E + \frac{p}{3\ep}
\end{equation}
Generalmente $E_m = 0$ cuando $E = 0$. Existen casos en los cuales $E = 0$ y $E_m = 0$ y esto se satisface para:
\begin{equation}
	p_o = 0, \qquad \frac{N \alpha}{3\ep} = 1.
\end{equation}
la cual es la condición de polarización permanente.






\chapter{Energía Electrostática}
Simplifica la resolución de algunos problemas. Y la contribución de energía de un sistema de cargas se divide en sus contribuciones cinética y potencial. \\
Energía potencial de un grupo de cargas puntuales
\begin{equation}
	u = \frac{1}{2} \sum _{j=1} ^{m} q_i \varphi _j.
\end{equation}
Energía electrostática de una distribució de carga:
\begin{align*}
	u &= \frac{1}{2} \ep \qty[\oint _S E\varphi \dd{a} + \int _V E^2 \dd{V}]. \\
	u &= \frac{1}{2} \int _V \rho (r) \varphi (r) \dd{V} + \frac{1}{2} \int _S \sigma (r) \varphi (r) \dd{a} + \frac{1}{2} \sum _j Q_j \varphi _j.
\end{align*}



\section{Condensadores}
Los condensadores son componentes eléctricos que sirven para almacenar energía. Dos conductores que puedan almacenar cargas iguales pero opuestas con una diferencia de potencial entre sí. La relación entre la carga almacenada y el potencial asociado es la capacitancia
\begin{equation}
	C \equiv \frac{Q}{\Delta \varphi}.
\end{equation}
Cuya energía se puede expresar como:
\begin{equation}
	u = \frac{1}{2} Q \Delta \varphi = \frac{1}{2} C \Delta \varphi ^2
\end{equation}
Si los conductores que forman un condensador tienen formas geométricas sencillas, la capacitancia puede obtenerse analíticamente. Y para condensadores en circuitos paralelos y en serie\footnote{En paralelo se tiene la misma diferencia de potencial entre los nodos y para los circuitos en serie se conserva la carga.}

\begin{align*}
	C_{eq} &= C_1 + C_2, \qquad \text{Paralelo} \\
	\frac{1}{C_{eq}} &= \frac{1}{C_1} + \frac{1}{C_2}, \qquad \text{Serie.}
\end{align*}



\chapter{Corriente Eléctrica}
Carga en movimiento $\to$ corriente eléctrica. Proceso por el cual se transporta la carga $\to$ condicción. La corriente se define como la velocidad a la que se transporta la carga a través de una superficie dad e un sistema conductor
\begin{equation}
	I = \dv{Q}{t}.
\end{equation}
Por convención: el sentido en que se mueve los portadores positivos se toma como el sentido de la corriente. \\

Cosas a notar:
\begin{itemize}
	\item En gases la conducción es más complicada, ya que las poblaciones d eportadores varían mucho con condiciones experimentales.
	\item Al estar en equilibrio térico cada partícula tiene un movimiento aleatorio.
	\item Líquidos y gases $\to$ movimientos hidrodinámicos.
\end{itemize}

\section{Densidad de Corriente}
Consideremos un medio conductor con un solo tipo de portador de carga:
\begin{itemize}
	\item Sea $N$ un número de portadores de carga por unidad de volumen.
	\item Asumimos una velocidad $v$ para los portadores de carga.
\end{itemize}
Se define la densidad de carga como
\begin{equation}
	\vec{J} = \sum _i N_i q_i v_i.
\end{equation}
Cuya integral para una superficie da como resultado la corriente
\begin{equation}
	I = -\oint _S \vec{J} \cdot \vu{n} \dd{a} = -\int _V \div{\vec{J}} \dd{V} = \int _V \pdv{\rho}{t} \dd{V}.
\end{equation}
Lo que implica y da como resultado la \textbf{ecuación de la continuidad} y representa la \textbf{conservación de la carga}
\begin{equation}
	\pdv{\rho}{t} + \div{\vec{J}} = 0.
\end{equation}


\section{Ley de Ohm}
Para mantener una corriente se requiere de un campo eléctrico, por lo que la densidad de corriente es proporcional a la fuerza por unidad de carga
\begin{align}
    \vec{J} &\propto f \\
    \vec{J} = gf &= g\frac{\vec{F}}{q} \\
    \vec{J} &= g\vec{E}.
\end{align}
Donde $g$ es la conductividad. La ley de Ohm des esta forma es válida en medios lineales isótropos o medios óhmicos. Generalmente se usa el recíproco de $g$:
\begin{equation}
    \sigma = \frac{1}{g}.
\end{equation}
A lo cual se conoce como resistividad. Para un cable recto conductor y, utilizando la definición de corriente, se tiene
\begin{equation}
    \Delta \varphi = IR
\end{equation}
Y sabiendo que $P = \dv{W}{t}$, entonces
\begin{equation}
    P = I\Delta \varphi = I^2 R = \qty(\frac{\Delta \varphi ^2}{R}).
\end{equation}

\section{Corrientes Estacionarias en Medios Continuos}
Consideremos un medio conductor óhmico (lineal), homogéneo, en condiciones de conducción en estado estacionario, es decir con densidad de cargo local en equilibrio:
\begin{equation}
    \pdv{\rho}{t} (x,y,z) = 0
\end{equation}
Por ende la ecuación de continuidad se reduce a:
\begin{equation}
    \div{\vec{J}} = \div{g\vec{E}} = 0
\end{equation}
Para un medio homogéneo, $g$ no depende de $E$:
\begin{align*}
    g\div{\vec{E}} &= 0 \\
    \div{\vec{E}} &= -\laplacian{\varphi} = 0
\end{align*}
Por lo que el problema de conducción en estado estacionario puede resolverse. Se resuelve utilizando la siguientes condiciones de frontera:
\begin{itemize}
    \item Para estado estacionario: $\mqty{\div{\vec{J}} = 0 \\ J_{1n} = J_{2n}}$.
    \item La componente normal debe ser continua $g_1 E_{1n} = g_2 E_{2n}$, análogo a $D_N$.
    \item Puesto que el campo eléctrico es estático en cada medio $\mqty{\oint \vec{E} \cdot \dd{l} \\ E_{1t} = E_{2n}}$
\end{itemize}
Otra relación entre conducción y electrostática se da considerando dos electrodos metálicos en un medio infinito óhmico homogéneo de conductividad $g$. Los electrodos están a $\varphi _1$ y $\varphi _2$ respectivamente:
\begin{equation}
    I = \frac{\varphi _1 - \varphi_2}{R}
\end{equation}
reemplazando la definición de corriente en términos del campo eléctrico, se tiene
\begin{equation}
    RC = \frac{\varepsilon}{g}.
\end{equation}

\subsection{Aproximación al Equilibrio Electrostático}
Para un conductor el equilibrio se alcanza en poco tiempo. Entre menos conductor el equilibrio se alcanza más lento. Consideremos un medio isótropo, homogéneo, caracterizado por $g$ y $E$. Con densidad volumétrica de carga $\rho _o (x,y,z)$. Si se aísla repentinamente de los campos aplicados
\begin{equation}
    \pdv{\rho}{t} + \frac{g}{\varepsilon} \rho = 0
\end{equation}
Resolvemos la ecuación diferencial, por lo que se tiene
\begin{equation}
    \rho = \rho _o e^{-\frac{g}{\varepsilon} t}.
\end{equation}


\section{Redes de Resistencias y Leyes de Kirchhoff}
Los portadores de carga siguen una trayectoria de baja resistencia llamada circuito. 
\begin{itemize}
    \item Las fuerzas puramente electrostáticas no puede hacer circular una corriente en el circuito.
    \item Se necesita una diferencia de potencial externa $\Delta \varphi$.
\end{itemize}
Para resistencias en serie y paralelo, se conserva la corriente y el potencial respectivamente, y con ello se tiene
\begin{align}
    R &= R_1 + R_2 \qquad \text{Serie} \\
    \frac{1}{R} &= \frac{1}{R_1} + \frac{1}{R_2} \qquad \text{Paralelo}.
\end{align}
Cualquier sistema puede resolverse de forma sistemática por medio de las leyes de Kirchhoff:
\begin{enumerate}
    \item La suma algebraica de las diferencias de voltaje en cualquier malla\footnote{Trayectoria cerrada conductora en un circuito.} de la red es cero.
        \begin{equation}
            \sum \varphi _i = 0.
        \end{equation}
    \item La suma algebraica de las corrientes que circulan hacia un nodo\footnote{Pounto donde concurren tres o más conductores.} es cero.
        \begin{equation}
            \sum I_i = 0.
        \end{equation}
\end{enumerate}
Interpretando las leyes de Kirchhoff
\begin{enumerate}
    \item Reafirmación de
        \begin{equation}
            \oint \vec{E} \cdot \dd{l} = 0.
        \end{equation}
    \item Enunciado formal de conservación de la carga
        \begin{equation}
            \div{\vec{J}} = 0.
        \end{equation}
\end{enumerate}


\section{Teoría Microscópica de la Conducción}
Consideremos una partícula libre de carga $q$ y masa $m$, bajo la influencia de una fuerza eléctrica local, su velocidad de deriva aumentará:
\begin{equation}
    m\dv{v}{t} = q\vec{E}.
\end{equation}
Si la partícula se encontrara en el vacío continuará acelerándose. En un medio material donde pasa una corriente constante la velocidad sería constante. Debe existir una fuerza debido al medio
\begin{equation}
    m\dv{v}{t} = q\vec{E} - Gv
\end{equation}
Para una $v$ constante: $v_d = \frac{g\vec{E}}{G}$. Y para el caso general
\begin{equation}
    v(t) = \frac{g}{G} E \qty(1 - e^{-\frac{Gt}{m}}).
\end{equation}
Para el estado estacionario: $v_d = q\frac{E}{G} = q\frac{\tau}{m} E$. Si hay varios tipos de portadores de carga
\begin{equation}
    g = \frac{\sum N_i q_i ^2 \tau _i}{m_i}
\end{equation}
donde $\tau$ es aproximadamente entre colisiones del electrón de conducción; por ende, el reocorrido libre medio del electrón es:
\begin{equation}
    l = v_t \tau.
\end{equation}
con $v_t \gg v_d$. Para los materiales de mayor conductividad eléctrica, sólo se considera un tipo de portador de carga, el electrón:
\begin{align*}
    J &= N_e e v_d \\
    g &= N_e e^2 \frac{\tau}{m} = N_e e \frac{v_d}{E}
\end{align*}
con $\dfrac{v_d}{E}$ es la movilidad del electrón.




\chapter{El Campo Magnético de Corrientes Estacionarias}
Así como las cargas generan campos eléctricos, las corrientes generan campos magnéticos. Para esta sección haremos la densidad de carga y el campo eléctrico igual a cero, así solo nos concentramos en el campo magnético.

\section{Ley de Ampére}
La primera ecuación de la magnetostática es
\begin{equation}
    \curl{\vb{B}} = \mu _o \vb{J}
\end{equation}
esta es conocida como la \textbf{ley de Ampére}. Para el caso integral, tomamos una superficie abierta $S$ con borde $\partial S$. Se puede utilizar el teorema de Stockes para convertir la integral del rotacional a una integral de línea
\begin{equation}
    \int _S \curl{\vb{B}} \cdot \dd{\vb{S}} = \oint _{\partial S} \vb{B} \dot \dd{\vb{r}} = \muc \int _S \vb{J} \cdot \dd{\vb{S}}.
\end{equation}
por lo que
\begin{equation}
    \oint _{\partial S} \vb{B} \cdot \vb{r} = \muc I.
\end{equation}

\section{Vector Potencial}
Para distribuciones de corrientes simples se sabe que se tiene
\begin{equation}
    \div{\vb{B}} = 0.
\end{equation}
Pero para corrientes más generales esto no es el caso. Para garantizar una solución a la ecuación anterior se escribe el campo magnético en términos de algún campo vectorial.
\begin{equation}
    \vb{B} = \curl{\vb{A}}.
\end{equation}
Utilizando propiedades de los operadores vectoriales, la ley de Ampére se vuelve
\begin{equation}
    \curl{\vb{B}} = -\laplacian{\vb{A}} + \grad{(\div{\vb{A}})} = \muc \vb{J}
\end{equation}

\subsection{Mono-polos Magnéticos}
Pasamos muy rápido de la ley de Ampére a su representación en términos del potencial magnético, que no paramos a analizar la situación. Supongamos una especie de carga puntual $g$ sería fuente de campo magnético
\begin{equation}
    \vb{B} = \frac{g\vu{r}}{4\pi r^2}
\end{equation}
a esto se le conoce como \textit{mono-polo magnético}. Las ecuaciones de Maxwell dicen que esto no existe. Pero, qué tan tajante es esta conclusión? Estamos seguros qué no existen? En caso de que existan las ecuaciones de Maxwell no cambiarían significativamente, aunque la idea de $\vb{A}$ se perdería, pero, es esto algo tan importante? Pues en Mecánica Cuántica esto no es posible, existen propiedades que se pierden al perder $\vb{A}$ (buscar Aharonov-Bohm Effect). \\

Y aún así, es posible que existan mono-polos magnéticos y tener una ''versión'' del vector $\vb{A}$ que permita la presencia de esta carga magnética, pero solo si esta esta relacionada a la carga del electrón de la siguiente forma
\begin{equation}
    ge = 2\pi \hbar n, \quad n \, \in \, \Z.
\end{equation}
Esta es conocida como \textit{La Condición de Cuantización de Dirac.}

\subsection{Transformaciones de Gauge}
La elección de $\vb{A}$ no es única: existen muchos de potenciales vectoriales que dan como resultado el mismo campo magnético. Esto es debido a que el rotacional del gradiente es cero. Esto significa que se puede agregar cualquier vector potencial de la forma $\nabla _\chi$ para alguna función $\chi$ y el campo magnético se mantiene igual
\begin{equation}
    \vb{A}^\prime = \vb{A} + \nabla _\chi \quad \Rightarrow \quad \curl{\vb{A}^\prime} = \curl{\vb{A}}.
\end{equation}
Un cambio del vector $\vb{A}$ como este se denomina \textit{transformación de gauge}.

\begin{teorema}
    Siempre se puede encontrar una transformación de auge $\chi$ tal que $\vec{A} ^\prime$ que satisface $\div{\vb{A}^\prime} = 0$. A esto se le conoce como \textit{Coulomb gauge}.
\end{teorema}
Si necesitamos $\div{\vb{A}^\prime} = 0$, solo necesitamos que nuestra transformación gauge obedesca
\begin{equation}
    \laplacian{\chi} = -\psi .
\end{equation}
La cual siempre tiene solución dado que es una ecuación de Poisson. Existe otra cantidad a la que se le conoce \textit{potencial escalar magnético}, $\Omega$. La idea detras del potencial es que podemos estar interesados en conocer el campo magnético en una región donde no haya corrientes y el campo eléctrico no cambie en el tiempo. En este caso debemos resolver $\curl{\vb{B}} = 0$, lo que se puede hacer teniendo
\begin{equation}
    \vb{B} = -\grad{\Omega}.
\end{equation}
Esto es solo válido en un número limitado de situaciones, esto gracias a la no existencia de cargas magnéticas.


\subsection{Ley de Biot-Savart}
Ahora utilizaremos el vector potencial para resolver para el campo magnético en la presencia de una distribución de corriente. Por ahora siempre diremos que trabajaremos en Coulomb gauge y nuestro vector potencial obedece que $\div{\vb{A}} = 0$. Entonces la ley de Ampére se vuelve lo siguiente
\begin{equation}
    \laplacian{\vb{A}} = -\muc \vb{J}.
\end{equation}
Utilizamos la solución más general, las funciones de Green, se tiene
\begin{equation}
    \vb{A} (x) = \frac{\muc}{4\pi} \int _V \dd ^3 x^\prime \qty(\frac{\vb{J} (\vx ^\prime)}{\abs{\vx - \vx ^\prime}})
\end{equation}
necesitamos recordar que el índice

\subsubsection{El Campo Magnético}
Del resultado anterior se tiene $\vb{B} = \curl{\vb{A}}$. Aplicando y recordando que $\nabla$ actúa sobre $\vx$. Encontramos
\begin{equation}
    \vb{B} (\vx) = \frac{\muc}{4\pi} \int _V \dd ^3 \vx ^\prime \frac{\vb{J} (\vx ^\prime) \times (\vx - \vx ^\prime)}{\abs{\vx - \vx^\prime}^3}
\end{equation}
Esta es conocida como la \textbf{Ley de Biot-Savart}. Esto describe el campo magnético de una densidad de corriente general. 



\section{Dipolos Magnéticos}
\subsection{Corriente en una Espira}
Empezamos por un ejemplo clásico, una espira. Encontramos el campo magnético generado por una espira, utilizamos $\vb{r}$ en lugar de $\vx$. El vector potencial está como
\begin{equation}
    \vb{A} (\vb{r}) = \frac{\muc}{4\pi} \int _V \dd ^3 r^\prime \frac{\vb{J} (\vb{r^\prime})}{\abs{\vb{r} - \vb{r}^\prime}}
\end{equation}
Ahora, reemplazando por la corriente, se tiene
\begin{equation}
    \vb{A} (\vb{r}) = \frac{\muc I}{4\pi} \oint _C \frac{\dd{\vb{r}}^\prime}{\abs{\vb{r} - \vb{r} ^\prime}}
\end{equation}
Pensando como sería esto muy lejos de la espira, podemos expandir en Taylor el término de la integral y esto nos quedaría
\begin{equation}
    \vb{A} (\vb{r}) = \frac{\muc I}{4\pi} \oint _C \vb{r}^\prime \qty(\frac{1}{r} + \frac{\vb{r} \cdot \vb{r}^\prime}{r^3} + \cdots).
\end{equation}
\subsubsection{Momento Dipolar Magnético}
Teniendo el vector de superficie
\begin{equation}
    \mathcal{\mathbf{S}} = \int _S \dd{\vb{S}}
\end{equation}
y dado que el primer término de la ecuación del potencial magnético se anula por la inexistencia de mono-polos, se tiene
\begin{equation}
    \vb{A} (\vb{r}) = \frac{\muc}{4\pi} \frac{\vb{m} \times \vb{r}}{r^3}
\end{equation}
y con esto, introducimos el \textit{momento dipolar magnético} 
\begin{equation}
    \vb{m} = I\mathcal{\mathbf{S}}.
\end{equation}
Y con un poco de álgebra se tiene
\begin{equation}
    \vb{B} (\vb{r}) = \frac{\muc}{4\pi} \qty(\frac{3(\vb{m} \cdot \vu{r}) \vu{r} - \vb{m}}{r^3}).
\end{equation}


\section{Fuerzas Magnéticas}
Hasta ahora vimos que las corrientes producen campos magnéticos. Y sabemos que una partícula cargada con una velocidad $\vb{v}$ experimentará una fuerza
\begin{equation}
    \vb{F} = q\vb{v} \times \vb{B}.
\end{equation}
Esto significa que si una segunda corriente es situada en algún lugar de la vecindad de la primera, esta experimentará una fuerza de la otra.

\subsection{Fuerza entre Corrientes}
Para dos corrientes paralelas se tiene que la fuerza que siente una de ellas debido a la otra es
\begin{equation}
    \vb{F} = q\vb{v} \times \vb{B} = q\vb{v} \times \qty(\frac{\muc I_1}{2\pi d}) \mathbf{\otimes}.
\end{equation}

Y la fuerza por unidad de longitud es
\begin{equation}
    \vb{f} = -\qty(\frac{\muc I_1 I_2}{2\pi d}) \vu{x}.
\end{equation}

Y, de la ley de Biot-Savart se concluye que la forma general para la fuerza entre corrientes es
\begin{equation}
    \vec{F} = \frac{\muc}{4\pi} I_1 I_2 \oint _{C_1} \oint _{C_2} \dd{\vb{r}_2} \times \qty(\dd{\vb{r}_1} \times \frac{\vb{r}_2 - \vb{r}_1}{\abs{\vb{r}_2 - \vb{r}_1}^3}).
\end{equation}
En general, esta integral es ''tricky'' de resolver, pero si las corrientes están bien separadas es más sencillo expresar la fuerza en términos del momento dipolar magnético.


\subsection{Fuerza y Energía de un Dipolo}
Para encontrar esto se parte de
\begin{equation}
    \vb{F} = \int _V \dd ^3 r \vb{J} (\vb{r}) \times \vb{B} (\vb{r})
\end{equation}
\dsnote{luego de mucho cálculo se llega a }
\begin{align}
    \vb{F} &= \curl{(\vb{B} \times \vb{m})}, \\
    \vb{F} = \grad{\vb{B} \cdot \vb{m}}.
\end{align}
Y sabiendo que toda fuerza se puede expresar como el gradiente de una función, la función del dipolo en el campo magnético es
\begin{equation}
    U = -\vb{B} \cdots \vb{m}.
\end{equation}


\subsection{Fuerza entre Dipolos}
Tomando el caso en el que el campo magnético lo genera un dipolo $\vb{m} _1$. Sabemos que el campo magnético es de la forma
\begin{equation}
    \vb{B} (\vb{r}) = \frac{\muc}{4\pi} \qty(\frac{3(\vb{m}_1 \cdot \vu{r}) \vu{r} - \vb{m}_1}{r^3})
\end{equation}
Y, utilizando lo anteriormente encontrado, se tiene que la fuerza que siente el segundo dipolo es
\begin{equation}
    \vb{F} = \frac{\muc}{4\pi} \grad{\qty(\frac{3(\vb{m}_1 \cdot \vu{r}) (\vb{m}_2 \cdot \vu{r}) - \vb{m}_1 \cdot \vb{m}_2}{r^3})}
\end{equation}
donde $\vb{r}$ es el vector de $\vb{m}_1$ a $\vb{m}_2$. Y, notemos que la estructura de la fuerza es idéntica a la que hay entre dos dipolos eléctricos. Esto es particularmente satisfactorio dado que se utilizan dos métodos diferentes.



\chapter{Propiedades Magnéticas de la Materia}
\section{Magnetización}
A diferencia de la polarización eléctrica que es casi siempre en dirección de $\vb{E}$, algunos materiales adquieren su magnetización paralela a $\vb{B}$ (\textbf{paramagnéticos}) y algunos opuesta a $\vb{B}$ (\textbf{diamagnéticos}); solo unas pocas sustancias (como el hierro) retienen su magnetización incluso después de que el campo externo sea removido, a estos se le conocen como \textbf{ferromagnéticos.}

\subsection{Torques y Fuerzas en Dipolos Magnéticos}
Así como los dipolos eléctricos sienten un torque, y esto sucede de igual forma para el campo magnético
\begin{equation}
	\vb{N} = \vb{m} \times \vb{B}.
\end{equation}
\dsnote{En Griffiths no se ve del todo bien esto, revisar la parte de torque magnético en Zemansky, está mas de ahuevo la figura ahí.}
Y para un loop infinitesimal
\begin{equation}
	\vb{F} = \grad{\vb{m} \cdot \vb{B}}.
\end{equation}


\subsection{Magnetización}
En presencia de un campo magnético la materia se \textbf{magnetiza}, por el análisis microscópico, se sabe que se tienen muchos minidipolos que resultan alineados en cierta dirección.
\begin{equation}
	\vb{M} = \text{ momento dipolar magnético por unidad de volumen.}
\end{equation}
En otras palabras
\begin{equation}
	\vb{M} = \frac{\Delta \vb{m}}{\Delta V}.
\end{equation}



\section{El Campo de un Objeto Magnetizado}
\subsection{Límites de Corrientes}
Suponga que se tiene una pieza de material magnetizado; su momento dipolar magnético por unidad de volumen $\vb{M}$. El vector potencial es
\begin{equation}
	\vb{A} (\vb{r}) = \frac{\muc}{4\pi} \int \frac{\vb{M} (\vb{r} ^\prime) \times \vu{r}}{r^2} \dd{\tau ^\prime}
\end{equation}
con $\dd{\tau ^\prime}$ es un diferencial de volumen. Reemplazando $\frac{\vu{r}}{r^2} = \nabla ^\prime (\frac{1}{r})$, lo que resulta en
\begin{equation}
	\vb{A} (\vb{r}) = \frac{\muc}{4\pi} \int \frac{1}{r} \qty[\nabla ^\prime \times \vb{M} (\vb{r}^\prime)] \dd{\tau ^\prime} + \frac{\muc}{4\pi} \oint \frac{1}{r} \qty[\vb{M} (\vb{r}^\prime \times \dd{a^\prime})]
\end{equation}
El primer término es el potencial de una corriente volumétrica
\begin{equation}
	\vb{J} _b = \curl{\vb{M}}
\end{equation}
y el segundo es el potencial de una corriente superficial
\begin{equation}
	\vb{K} _b = \vb{M} \times \vu{n}.
\end{equation}
Esto es el análogo a la carga de volumen y superficie de polarización $\rho _b = -\div{\vb{P}}$ y $\sigma _b = \vb{P} \cdot \vu{n}$. Con todo esto, podemos expresar el campo magnético como
\begin{equation}
	\vb{B} (\vb{r}) = -\muc \grad{\varphi (\vb{r})} + \muc \vb{M} (\vb{r})
\end{equation}
donde $\varphi (\vb{r})$ es el campo escalar magnético \dsnote{Anteriormente lo introduje como $\Omega$, pero cambié de libro.} Y tiene la siguiente forma
\begin{equation}
	\varphi (\vb{r}) = \frac{1}{4\pi} \int _{V_o} \vb{M} (\vb{r} ^\prime) \cdot \frac{\vb{r} - \vb{r} ^\prime}{\abs{\vb{r} - \vb{r} ^\prime}^3} \dd{v} ^\prime.
\end{equation}

\subsection{Potencial Escalar Magnético y Densidad de Polos Magnéticos}
La expresión del potencial escalar magnético, es de forma parecida a la del potencial electrostático que provioene de un material dieléctrico polarizado. Utilizando la ecuación
\begin{equation}
	\frac{\vb{M} \cdot (\vb{r} - \vb{r}^\prime)}{\abs{\vb{r} - \vb{r} ^\prime}^3} = \vb{M} \cdot \nabla ^\prime \frac{1}{\abs{\vb{r} - \vb{r} ^\prime}} = \nabla ^\prime \cdot \frac{\vb{M}}{\abs{\vb{r} - \vb{r} ^\prime}} - \frac{1}{\abs{\vb{r} - \vb{r} ^\prime}} \nabla ^\prime \cdot \vb{M} .
\end{equation}
Y el potencial se reescribe como
\begin{equation}
	\varphi (\vb{r}) = \frac{1}{4\pi} \int _{S_o} \frac{\vb{M} \cdot \vu{n} \dd{a^\prime}}{\abs{\vb{r} - \vb{r}^\prime}} - \frac{1}{4\pi} \int _{V_o} \frac{\nabla ^\prime \cdot \vb{M}}{\abs{\vb{r} - \vb{r}^\prime}} \dd{v^\prime}.
\end{equation}
Con lo que se tiene
\begin{description}
	\item[Densidad de Polos Magnéticos: ] $\rho _M = -\nabla ^\prime \cdot \vb{M} (\vb{r} ^\prime)$.
	\item[Densidad Superficial de la Intensidad de Polos Magnéticos: ] $\sigma _M (\vb{r} ^\prime) = \vb{M} (\vb{r}^\prime) \cdot \vu{n}$.
\end{description}

\section{Fuentes de Campo Magnético: Intensidad Magnética}
Así como se tiene el desplazaiento eléctrico, definimos la \textbf{intensidad magnética} $\vb{H}$ como
\begin{equation}
	\vb{H} = \frac{1}{\muc} \vb{B} - \vb{M}.
\end{equation}
Combinando ecuaciones se tiene
\begin{equation}
	\vb{H} (\vb{r}) = \frac{1}{4\pi} \int _V \frac{\vb{J} \times (\vb{r} - \vb{r} ^\prime)}{\abs{\vb{r} - \vb{r} ^\prime}^3} \dd{v}^\prime - \grad{\varphi} (\vb{r}).
\end{equation}
Parece que no hemos ganado nada, dado que $\vb{H}$ aún depende de $\vb{M}$ a través de $\rho _m$ y $\sigma _m$. Más adelante se verá la relación entre esta nueva cantidad y la densidad de corriente. 




\chapter{Inducción Electromagnética}














\chapter{Energía Magnética}













\chapter{Corrientes que Varían Lentamente}















\chapter{Ecuaciones de Maxwell}














\chapter{Propagaricón de Ondas Monocromáticas}










































%%%%%%%%%%%%%%%%%