\part{Electromagnetismo}

\vspace*{\fill}

\begin{center}
	\textit{''El conocimiento es una red de ideas interconectadas, una vez que atrapamos una, las demás vienen detrás'' - Michael Faraday.}
\end{center}

\vspace*{\fill}

\chapter{Análisis Vectorial}
\dsnote{Solo se enunciarán los teoremas y propiedades más importantes.}
\section{Operadores Vectoriales}
\subsection{Gradiente}
El gradiente de un campo escalar es un campo vectorial. Es un vector normal a la curva de nivel en el punto estudiado
\begin{equation}
	\grad{f(r)} = \qty(\pdv{f}{x_1} \vu{x}_1 + \cdots + \pdv{f}{x_n} \vu{x}_n),
\end{equation}
Cuando $\grad{f} = 0$ se tiene un punto estacionario (máximo o mínimo). Dicho de otra forma, el gradiente marca la dirección en la cual varía más rápido el campo escalar y su magnitud indica cuan rápido varía. \\
\subsubsection{En Diferentes Sistemas de Coordenadas}
\begin{align*}
	\grad{\varphi} &= \pdv{\varphi}{r} \vu{r} + \frac{1}{r} \pdv{\varphi}{\theta} \vu{\theta} + \frac{1}{r\sin{\theta}} \pdv{\varphi}{\phi} \vu{\phi} \qquad \text{Coordenadas Esféricas} \\
	\grad{\varphi} &= \pdv{\varphi}{\rho} \vu{\rho} + \frac{1}{\rho} \pdv{\varphi}{\phi} \vu{\phi} + \pdv{\varphi}{z} \vu{z}.
\end{align*}
 


\subsection{Divergencia}
La divergencia de un campo vectorial mide la diferencia entre el flujo saliente y entrante en una región dada.
\begin{equation}
	\div{A} = \pdv{A_x}{x} + \pdv{A_y}{y} + \pdv{A_z}{z}.
\end{equation}
La divergencia produce un escalar, mide cuanto un vector se dispersa o sale de un punto.
\subsubsection{En Diferentes Sistemas de Coordenadas}
\begin{align*}
	\div{A} &= \frac{1}{r^2} \pdv{(r^2 A_r)}{r} + \frac{1}{r\sin{\theta}} \pdv{(\sin{\theta} A_\theta)}{\theta} + \frac{1}{r\sin{\theta}} \pdv{A_\phi}{\phi} \qquad \text{Coordenadas Esféricas} \\
	\div{A} = \frac{1}{\rho} \pdv{(\rho A_\rho)}{\rho} + \frac{1}{\rho} \pdv{A_\phi}{\phi} + \pdv{A_z}{z} \qquad \text{Coordenadas Cilíndricas.}
\end{align*}








\subsection{Rotacional}
Operador vectorial sobre campos vectoriales: tendencia de un campo vectorial a inducir rotación.
\begin{equation}
	\curl{A} = \mqty| \vi & \vj & \vk \\ \pdv{x} & \pdv{y} & \pdv{z} \\ A_x & A_y & A_z |
\end{equation}


\section{Integración Vectorial}
Integrales de línea, superficie y volumen. Integrandos pueden ser vectores o escalares. \\
\subsection{Integrales de Línea}
Recorridos o trayectorias
\begin{equation}
	\int _c F(r) \cdot \dd{l}.
\end{equation}
Donde $F$ es un campo vectorial, $\dd{l}$ desplazamiento vectorial infinitesimal a lo largo de la curva y $c$ curva sobre la cual se integra. Cuando la integral de línea es independiente de la trayectoria el campo es conservativo.

\subsection{Integrales de Superficie}
Esta integral mide un flujo
\begin{equation}
	\int _S F\cdot \vu{n} \dd{a}
\end{equation}
Esto nos da un escalar. Si laintegral es cerrada $\oint F\cdot \vu{n} \dd{a}$.

\subsection{Teoremas Importantes}
\begin{teorema}
	\textbf{Teorema Fundamental de la Divergencia: }
	\begin{equation}
		\int _V \div{F} \dd{V} = \oint _S F\cdot \vu{n} \dd{a}.
	\end{equation}
\end{teorema}


\begin{teorema}
	\textbf{Teorema de Stokes: }
	\begin{equation}
		\int _S (\curl{F}) \cdot \vu{n} \dd{a} = \oint _c F \cdot \dd{l}.
	\end{equation}
\end{teorema}




\chapter{Electrostática}
Esta estudia los efectos mutuos que se producen entre los cuerpos como consecuencia de us carga eléctrica. La carga eléctrica es una propiedad fundamental y característica delas partículas elementales.
\begin{enumerate}
	\item La carga no se crea ni se destruye.
	\item Hay dos clases: positiva y negativa.
	\item En un sistema cerrado la carga se conserva.
\end{enumerate}

\section{Ley de Coulomb}
Experimentalemnte
\begin{enumerate}
	\item Dos cargas puntuales ejercen fuerza a lo largo de la línea que las une e inversamente proporcional al cuadrado de la distancia.
	\item La fuerza es proporcional al producto de las cargas.
\end{enumerate}

\begin{equation}
	\vec{F} = \frac{1}{4\pi \varepsilon _o} \frac{q_1 q_2}{r_{12} ^2} \vu{r}_{12}.
\end{equation}
Esto es válido para cargas puntuales, pero para muchas cargas se aplica el principio de superposición. Ahora, si la carga se distribuye en un volumen $V$ con densidad $\rho$ y sobre la superficie $S$ que limita $V$ con densidad $\sigma$, la fuerza ejercida por esta distribución de carga sobre una carga puntual $q$ está dada por:
\begin{align}
	F(r) &= \frac{q}{4\pi \ep} \int _V \rho (r') \frac{r - r'}{\abs{r - r'}^3} \dd{V'} + \frac{q}{4\pi \ep} \int _S \sigma (r') \frac{r - r'}{\abs{r - r'}^3} \dd{S'}.
\end{align}


\section{Campo Eléctrico}
Campo vectorial físico $\to$ región del espacio en la que interactúa la fuerza eléctrica. Se genera por cargas o campos magnéticos variables en el tiempo.
\begin{equation}
	\vec{E} (r) = \frac{1}{4\pi \ep} \sum _{i=1} ^N \frac{q_i}{r_i ^2} \vu{r} _i.
\end{equation}
Entonces, de manera general, tomando en cuenta distribuciones continuas de carga y cargas puntuales
\begin{equation}
	\vec{E} (r) = \kel \sum _{i=1} ^N q_i \frac{r - r_i}{\abs{r - r_i}^3} + \kel \int _V \frac{r - r'}{\abs{r - r'}^3} \rho (r') \dd{V'} + \kel \int _S \frac{r - r'}{\abs{r - r'}^3} \sigma (r') \dd{a'}
\end{equation}



\section{Potencial Electrostático}
El campo eléctrico cumple con
\begin{equation}
	\curl{\vec{E}} = 0.
\end{equation}
Sabemos que $\vec{E} (r) = -\grad{\varphi}$ es un potencial electrostático. Pero con esto se tiene
\begin{equation}
	\varphi (r) = -\int _{\mathcal{O}} ^r \vec{E} \cdot \dd{\vec{l}}
\end{equation}
donde $\mathcal{O}$ es el punto de referencia. Notemos que

\begin{enumerate}
	\item Potencial electrostático no es lo mismo que energía potencial.
	\item $\vec{E}$ es un vector derivado de un escalar y ya que $\curl{E} = 0$ esto implica: las componentes de $\vec{E}$ no son independientes
	\begin{equation}
		\pdv{E_x}{y} = \pdv{E_y}{x}, \qquad \pdv{E_z}{y} = \pdv{E_y}{z}, \qquad \pdv{E_x}{z} = \pdv{E_z}{x}.
	\end{equation}
	\item EL sistema de referencia es fundamental.
	\item El potencial obedece el principio de superposición.
	\begin{equation}
		\varphi (r) = \kel \sum_{i=1} ^N \frac{q_i}{\abs{r - r_i}} + \kel \int _V \frac{\rho (r')}{\abs{r-r'}} \dv{V'} + \kel \int _S \frac{\sigma (r')}{\abs{r - r'}} \dd{a'}.
	\end{equation}
\end{enumerate}

\section{Ley de Gauss}
Importante relación entre la integral de la componente normal del campo eléctrico sobre una superficie cerrada y la carga total encerrada por la superficie.
\begin{equation}
	\Psi _E = \int _S \vec{E} \cdot \vu{n} \dd{a}, \qquad \text{Flujo Eléctrica}
\end{equation}
Para una superficie cerrada imaginaria (superficie gaussiana). Se tiene que $\Psi _E \propto q$. Entonces
\begin{equation}
	\oint \vec{E} \cdot \vu{n} \dd{a} = \frac{1}{\ep} Q_{enc}.
\end{equation}
y por el teorema de la divergencia
\begin{equation}
	\div{\vec{E}} = \frac{\rho}{\ep}.
\end{equation}
A notar:
\begin{enumerate}
	\item Simetría esférica: Superficie gausiana $\to$ esfera concéntrica.
	\item Simetría cilíndrica: Superficie gaussiana $\to$ cilindro coaxial.
	\item Simetría palna: Superficie gaussiana $\to$ caja.
\end{enumerate}



\section{Dipolo Eléctrico}
Dos cargas iguales de signo contrario separados por una pequeña distancia. Supongamos una carga $-q$ en $r'$ y una carga $q$ en $r' + l$. Entonces
\begin{equation}
	\vec{E} (r) = \kel q \qty(\frac{r - r' - l}{\abs{r - r' - l}^3} - \frac{r - r'}{\abs{r - r'}^3})
\end{equation}
expandiendo el término de dentro
\begin{equation}
	\vec{E} (r) = \kel q \qty[\frac{3 (r - r') l}{\abs{r - r'}^3} (r - r') - \frac{l}{\abs{r - r'}} + \cdots]
\end{equation}

\subsection{Momento Dipolar Eléctrico}
Si se coloca un dipolo en un campo $\vec{E}$, ambas cargas $(q,-q)$ separadas una distancia $l$, experimentan fuerzas de igual magnitud y dirección contraria $\vec{F}$ y $-\vec{F}$, por ende: $\sum vec{F} = 0$ y $\sum \tau \neq 0$. Definimos el \textbf{momento dipolar} como $\vec{p} = q\vec{l}$. Por ende
\begin{align}
	\vec{E} (r) &= \kel \qty[\frac{3(r - r')\cdot \vec{p}}{\abs{r - r'}^3} (r - r') - \frac{\vec{p}}{\abs{r - r'}^3}].	\\
	\varphi (r) &= \kel q \qty[\frac{1}{\abs{r - r' - l}} - \frac{1}{\abs{r - r'}}] = \kel \frac{(r - r') \cdot \vec{p}}{\abs{r - r'}^3}.
\end{align}


\section{Trabajo y Energía en Electrostática}
Suponga una configuración de cargas estacionarias y se desea mover una carga de prueba $Q$ de un punto $a$ a un punto $b$. Por ende
\begin{equation}
	\varphi (b) - \varphi (a) = \frac{W}{Q}
\end{equation}
La diferencia de potencial entre $a$ y $b$ es igual al trabajo por unidad de carga requerido para mover $Q$ de $a\to b$.

\subsection{Energía de una Distribución de Cargas Puntuales}
Energía de la distribución $\to$ trabajo para ensamblar la distribución.
\begin{equation}
	W = \frac{1}{2} \sum_{i=1} ^n q_i \varphi (r_i).
\end{equation}
de otra forma se puede reescribir el trabajo
\begin{align}
	W &= \frac{\ep}{2} \qty[\oint _S \varphi \vec{E} \cdot \dd{a} + \int _V E^2 \dv{V}] \\
	W &= \frac{\ep}{2} \int _V E^2 \dv{V} \qquad \text{Si el volumen crece } (r\to \infty).
\end{align}





\chapter{Problemas Electrostáticos}

Estos problemas no son secillos de resolver. Por otro lado:
\begin{equation}
	\laplacian{\varphi} = -\frac{\rho}{\ep} \qquad \text{Ecuación de Poisson}.
\end{equation}

\textbf{Laplaciano:}
\begin{enumerate}
	\item \textbf{Rectangulares:}
	\begin{equation}
		\laplacian{\varphi} = \pdv[2]{\varphi}{x} + \pdv[2]{\varphi}{y} + \pdv[2]{\varphi}{z}.
	\end{equation}
	\item \textbf{Esféricas: }
	\begin{equation}
		\laplacian{\varphi} = \frac{1}{r^2} \pdv{r} \qty(r^2 \pdv{\varphi}{r}) + \frac{1}{r^2 \sin{\theta}} \pdv{\theta} \qty(\sin{\theta} \pdv{\varphi}{\theta}) + \frac{1}{r^2 \sin{\theta}} \pdv[2]{\varphi}{\phi}.
	\end{equation}
	\item \textbf{Cilíndricas: }
	\begin{equation}
		\laplacian{\varphi} = \frac{1}{\rho} \pdv{\rho} \qty(\rho \pdv{\varphi}{\rho}) + \frac{1}{\rho ^2} \pdv[2]{\varphi}{\theta} + \pdv[2]{\varphi}{z}.
	\end{equation}
\end{enumerate}

Y cuando interesa conocer el potencial en regones con $\rho = 0$.
\begin{equation}
	\laplacian{\varphi} = 0 \qquad \text{Ecuación de Laplace.}
\end{equation}

\section{Ecuación de Laplace}
\begin{teorema}
	Si $\varphi _1 ,\ldots ,\varphi _n$ son todas soluciones de la ecuación de Laplace, entonces:
	\begin{equation}
		\varphi = c_1 \varphi _1 + \cdots + c_n \varphi _n,
	\end{equation}
	con $c_i = $ ctes, también es solución de la ecuación de Laplace
	\begin{equation}
		\laplacian{\varphi} = c_1 \laplacian{\varphi}_1 + \cdots + c_n \laplacian{\varphi}_n = 0.
	\end{equation}
\end{teorema}


\begin{teorema}
	Dos sluciones de la ecuación de Laplace que satisfacen las mismas condiciones de frontera difieran a lo sumo en una constante aditiva.
\end{teorema}


\subsection{Ecuación de Laplace en una Dimensión}
Si $\varphi$ es una función de una variable. Las soluciones para cada uno de los sistemas importantes de coordenadas

\begin{enumerate}
	\item \textbf{Rectangulares:}
	\begin{equation}
		\varphi (x) = ax + b.
	\end{equation}
	\item \textbf{Esféricas: }
	\begin{equation}
		\varphi (r) = -\frac{a}{r} + b.
	\end{equation}
	\item \textbf{Cilíndricas: }
	\begin{equation}
		\varphi = a\ln{\abs{\rho}} + b.
	\end{equation}
\end{enumerate}

\subsubsection{Caso Esférico}
Para el caso esférico en dos dimensiones su solución luego de la separación de variables es
\begin{equation}
	\varphi (r,\theta) = \sum _{n=0} ^\infty \qty(A_n r^n + \frac{B_n}{e^{n+1}}) P_n (\cos{\theta}).
\end{equation}

\subsubsection{Caso Cilíndrico}
Ahora para el caso cilíndrico en dos dimensiones, su solución es:
\begin{equation}
	\varphi (\rho ,\theta) = A_o + B_o \ln{\abs{\rho}} + \sum _{n=1} ^\infty \qty(A_n \rho ^n + B_n \rho ^{-n}) \qty(C_n \cos{n \theta} + D_n \sin{n\theta}).
\end{equation}

\subsubsection{Caso Cartesiano}
Ahora la solución del caso cartesiano
\begin{equation}
	\varphi (x,y,z) = A e^{-(k + m) ^{1/2} x} \cos{m^{1/2} y} \cos{k^{1/2} z}.
\end{equation}

\section{Imágenes Electrostáticas}
Para un conujunto dado de condicioens de frontera, la solución a la ecuación de Laplace es única y resolviendo para $\varphi$ se ha encontrado la solución completa al problema. \\
El método de imágenes es un procedimiento para lograr este resultado sin resolver específicamente la ecuacioń diferencial. \dsnote{No se aplica universalmente, solo un número considerable de problemas (que bueno xd).} \\

Supongamos una carga $q$ arriba de un plano conductor: La carga $q$ inducirá una carga en el conductor. La potencia total a una distancia $\vec{r}$ estará dado por la contribución de la carga más la inducida en el conductor:
\begin{align*}
	\varphi (r) &= \varphi _1 (r) + \kel \int _S \frac{\sigma (r')}{\abs{r - r'}} \dd{a'}. \\
	\varphi (r) &= \varphi _1 (r) + \varphi _2 (r).
\end{align*}
$\varphi _2 (r)$ puede ser sustituído por un potencial debido a una distribución de carga especificada: 
\begin{itemize}
	\item Carga Imagen.
	\item Debe cumplir con las condiciones de frontera.
\end{itemize}


\subsection{Sistemas Conductores y Coeficientes de Potencial}
Cuando se tienen conductores con formas complicadas las soluciones analíticas quedan descartadas: Métodos numéricos. Se pueden sacar algunas conclusiones, supongamos $N$ conductores en una geometría fija:
\begin{itemize}
	\item Existe una relación lineal entre el potencial de un conductor y las cargas de los diversos conductores del sistema. $N$ conductores descargados excepto el conductor $j\to$ carga $Q_{jo}$. La solución de Laplace es el espacio fuera de los conductores la expresamos como:
	\begin{equation}
		\varphi _1 ^{(j)} (x,y,z) \quad \to \quad \text{potencial generado por } Q_{jo}.
	\end{equation}
	El potencial de cada uno de los conductores estará dado por:
	\begin{equation}
		\varphi _1 ^{(j)}, ldots, \varphi _N ^{(j)}
	\end{equation}
	Reexpresando la carga del $j-$ésimo
	$$ \lambda Q_{jo}, \quad \lambda = \text{cte}, \quad \text{sadisface la ecuación de Laplace.} $$
	\item El potencial también se multiplica por $\lambda $.
	\item Todas las derivadas se multiplica por $\lambda$.
	\item $\sigma 0 -\ep \pdv{\varphi}{\vu{n}} \to$ todas las densidades se multiplican por $\lambda$.
	\item Potencial de cada conductor es proporcional a $Q_j$
	\begin{equation}
		\varphi _i ^j = p_{ij} Q_j, \qquad (i = 1,2,\ldots ,M)
	\end{equation}
	donde $p_{ij}$ es constante que depende de la geometría utilizando el mismo argumento para conductor $k$ $Q_k = \nu Q_{k_o}$ con $\nu =$ cte. Generalizando
	\begin{equation}
		\varphi _i = \sum _{j=1} ^{M} p_{ij} Q_j.
	\end{equation}
\end{itemize}









\chapter{Campo Electrostático en Medios Dieléctricos y su Teoría Microscópica}
Existen materiales conductoresy aislantes (dieléctricos)
\begin{itemize}
	\item Material dieléctrico ideal: material que no tiene cargas libres.
	\item Bajo la presencia de un campo eléctrico externo: pueden tener pequeños desplazamientos.
	\item Dieléctrico se polariza. Un dieléctrico polarizado es eléctricamente neutro, sin embargo produce un campo eléctrico en los puntos exteriores en interiores del dieléctrico. En un caso extermo, bajo un campo muy grande, ocurre la ionización. Un dipolo inducido $\vec{p} = \alpha \vec{E}$ donde $\alpha$ es la polarizabilidad atómica, depende de la estructura del átomo. Para moléculas es más complejo debido a que tienen direcciones preferenciales $\alpha \to \alpha _{ij}$.
\end{itemize}

\section{Polarización}
Consideremos un pequeño volumen $\Delta V$ de un medio dieléctrico que como todo es eléctricamente neutro. Si el medio está polarizado: Momento dipolar eléctrico
	\begin{equation}
		\Delta \vec{p} = \int _{\Delta V} r\dd{q}.
	\end{equation}
$\Delta \vec{p}$ depende del tamao del elemento de volumen. Se introduce el momento dipolar eléctrico por unidad de volumen y se conoce como polarización eléctrica:
\begin{equation}
	P \equiv \frac{\Delta \vec{p}}{\Delta V}.
\end{equation}
Momento dipolar de una molécula
\begin{equation}
	\vec{p} _m = \int _{\text{milecula}} r\dd{q}.
\end{equation}



\subsection{Campo Fuera de un Medio Dieléctrico}
Consideramos una porción finita de material dieléctrico polarizado. Esta polarización genera un campo eléctrico $\vec{E} \to \varphi$. Cada elemento de $\Delta V'$ se caracteriza por un momento dipolar $\Delta \vec{p} = P\Delta V'$ y tenemos $r \gg r'$. La contribución de las cargas en $\Delta V'$ al potencial está dada por:
\begin{equation}
	\Delta \varphi (r) = \kel \frac{\Delta p \cdot (r - r')}{\abs{r - r'}^3} = \kel \frac{P(r') \cdot (r - r') \Delta V'}{\abs{r - r'}^3}.
\end{equation}
El potencial total es la suma de todas las contribuciones de todas las partes del dieléctrico:
\begin{equation}
	\varphi (r) = \kel \int _{V_o} \frac{P(r') (r - r')}{\abs{r - r'}^3} \dd{V'}.
\end{equation}
Ahora buscamos encontrar $\vec{E}$
\begin{equation}
	\frac{r - r'}{\abs{r - r'}^3} = \nabla ^\prime \qty(\frac{1}{\abs{r - r'}}).
\end{equation}
Desarrollando la siguiente identidad: $\nabla ^\prime (f\vec{F}) = f\nabla ^\prime \cdot \vec{F} + \vec{F} \cdot \nabla ^\prime f.$ Se desarrolla la integral anterior, con ello se llega a 
\begin{equation}
	\varphi (r) = \kel \oint _{S_o} \frac{P \cdot \vu{n} \dd{a}}{\abs{r - r'}} + \kel \int _{V_o} \frac{(-\nabla ^\prime \cdot P) \dd{V}}{\abs{r - r'}}
\end{equation}
Con lo que se definen dos nuevas funciones escalares
\begin{itemize}
	\item $\sigma _p = P \cdot \vu{n}$.
	\item $\rho _p = -\div{P}$.
\end{itemize}

\begin{itemize}
	\item La densidad superficial de carga de polarización está dada por la componente de polarización normal a la superficie.
	\item La densidad de carga de polarización volumétrica es una medida de la no uniformidad de la polarización dentro del material.
\end{itemize}

Luego, calculando el campo eléctrico
\begin{equation}
	\vec{E} = \kel \qty[\int _S \sigma _p \frac{r - r'}{\abs{r - r'}^3} \dd{a'} + \int _{V_o} \rho _p \frac{r - r'}{\abs{r - r'}^3} \dd{V'}].
\end{equation}


\subsection{Campo Eléctrico dentro de un Dieléctrico}
En un dieléctrico la carga de prueba es comparable al tamaño de las moléculas. EL campo eléctrico dentro del dieléctrico debe tener las mismas propiedades. El campo eléctrico en un dieléctrico es igual al campo eléctrico dentro de una cavidad. \\

La ley de Gauss en un dieléctrico viene dada por
\begin{equation}
	\oint (\ep \vec{E} + \vec{P}) \cdot \vu{n} \dd{a} = Q.
\end{equation}
donde el término $\vec{D} = \ep \vec{E} + \vec{P}$ se le denomina \textbf{desplazamiento eléctrico}. 


\subsection{Suceptibilidad Dieléctrica y Constante Dieléctrica}
La polarización de un medio dieléctrico ocurre en respuesta al campo eléctrico en el medio. El grado de polarización depende:
\begin{itemize}
	\item Campo eléctrico
	\item Propiedades del material
\end{itemize}
A nivel macroscópico $F = P(E)$. En la mayoría de materiales $P$ se anula cuando $\vec{E} = 0$. Para materiales de este tipo y si son materiales siótropos, la polarización tendrá el mismo sentido que $\vec{E}$. 
\begin{equation}
	P = \chi (E) \vec{E} = \ep \chi _e \vec{E}.
\end{equation}
con esto se define $\varepsilon = \ep (1 - \chi _e)$.


\section{Condiciones de Frontera para los Vectores de Campo}
Variación de $\vec{E}$ y $\vec{D}$ al pasar por una zona interfacial entre dos medios. Considerando dos emdios encontacto y una densidad superficial de carga externa $\sigma$. Construir una pequeña superficie $S$: forma de caja de pastillas de altura despreciable. Entonces
\begin{equation}
	D_{2n} - D_{1n} = \sigma
\end{equation}
Observaciones
\begin{itemize}
	\item La discontinuidad en la componente normal $\vec{D}$ está dada por la densidad superficial de carga en la zona interfacial.
	\item Si no hay carga en la zona interfacial la componente normal de $\vec{D}$ no es contínua.
\end{itemize}
Y por la ley de Gauss se concluye que la componente tangencial del campo eléctrico es continua.
\dsnote{Después de toda esta parafernalia, el ejemplo clásico es el de la esfera dieléctrica en un campo eléctrico extermo, revisar el libro.}



\section{Teoría Microscópica de los Dieléctricos}
\subsection{Campo Molecular en un Dieléctrico: $E_m$}
Es el campo eléctrico en una posición molecular del dieléctrico el cual es producido por todas las fuentes externas y por todas las moléculas polarizadas del dieléctrico con excepción de la molécula en el punto considerado. El dieléctrico se polariza al inducir un campo. Suponemos polarización uniforma $\div{\vec{P}} = 0$. El campo eléctrico en el centro de la cavidad puede expresarse como:
\begin{equation}
	\vec{E}_m = \vec{E} _x + \vec{E}_d + \vec{E} _s + \vec{E} ^\prime
\end{equation}

\begin{itemize}
	\item $E_x$ campo eléctrico primario debido a los planos.
	\item $E_d$ campo debido a la carga de polarización e la superficie.
	\item $E_s$ campo debido a la carga de polarización en la superficie $S$.
	\item $E^\prime$ campo generado por dipolos dentro de $S$.
\end{itemize}


\subsection{Moléculas Polares}
\begin{itemize}
	\item Momento dipolar permanente
	\item Están formadas por al menos dos especcies distintas de átomos.
	\item En ausencia de campo eléctrico por una porción macroscópica del dieléctrico polar no está polarizada: dipolos individuales orientados al azar.
\end{itemize}
Si el dieléctrico polar se somete a un campo eléctrico, los dipolos se alinean con el campo. Si el campo es lo suficientemente intenso, la polarización alcanza el vapor de saturación: 
\begin{equation}
	P_s = N\vec{p}_m
\end{equation}
Se requiere valores de campo muy intensos. Si la temperatura se eleva la polarización disminuye. Según la mecánica estadística, a una temperatura $T$, la probabilidad de encontrar una molécula con energía $E$ es:
\begin{equation}
	f(E) = \propto e^{-E/kT}
\end{equation}
La energía potencial de un dipolo permanente $p_o$ en un campo eléctrico es:
\begin{equation}
	u = -p_o \cdot E_m
\end{equation}
La energía cinética de las moléculas no dependen del campo, entonces se desprecia su contribución en la distribución. El moemtno diplar efectivo de un dipolo molecular es su componente en la dirección del campo: $p_o \cos{\theta}$. El valor promedio de la cantidad está dado por:
\begin{equation}
	\expval{x}  =\sum x_m p_m = \frac{\sum _m \chi _m e^{-\beta e_m}}{\sum _m e^{-\beta u_m}}.
\end{equation}

pasando a lo continuo
\begin{equation}
	\expval{p_o \cos{\theta}} = p_o \qty(-\frac{1}{y} + \coth{y})
\end{equation}
con $y = p_o \frac{E_m}{kT}$ y a esta fórmula se le conoce como \textbf{Fórmula de Lagevin}. y el momento dipolar efectivo promedio
\begin{equation}
	\expval{p_o \cos{\theta}} = \frac{p_o ^2 E_m}{3kT}.
\end{equation}
Polarizabilidad por orientación $\alpha = \frac{p_o ^2}{3kT}$.

\begin{equation}
	\alpha = \alpha _o + \frac{p_o ^2}{3kT}
\end{equation}
a esta se le conoce como la \textbf{ecuación de Langevin-Debye}.

\subsection{Polarización Permanente: Ferroelectricidad}
Sabemos que
\begin{equation}
	E_m = E + \frac{p}{3\ep}
\end{equation}
Generalmente $E_m = 0$ cuando $E = 0$. Existen casos en los cuales $E = 0$ y $E_m = 0$ y esto se satisface para:
\begin{equation}
	p_o = 0, \qquad \frac{N \alpha}{3\ep} = 1.
\end{equation}
la cual es la condición de polarización permanente.






\chapter{Energía Electrostática}
Simplifica la resolución de algunos problemas. Y la contribución de energía de un sistema de cargas se divide en sus contribuciones cinética y potencial. \\
Energía potencial de un grupo de cargas puntuales
\begin{equation}
	u = \frac{1}{2} \sum _{j=1} ^{m} q_i \varphi _j.
\end{equation}
Energía electrostática de una distribució de carga:
\begin{align*}
	u &= \frac{1}{2} \ep \qty[\oint _S E\varphi \dd{a} + \int _V E^2 \dd{V}]. \\
	u &= \frac{1}{2} \int _V \rho (r) \varphi (r) \dd{V} + \frac{1}{2} \int _S \sigma (r) \varphi (r) \dd{a} + \frac{1}{2} \sum _j Q_j \varphi _j.
\end{align*}



\section{Condensadores}
Los condensadores son componentes eléctricos que sirven para almacenar energía. Dos conductores que puedan almacenar cargas iguales pero opuestas con una diferencia de potencial entre sí. La relación entre la carga almacenada y el potencial asociado es la capacitancia
\begin{equation}
	C \equiv \frac{Q}{\Delta \varphi}.
\end{equation}
Cuya energía se puede expresar como:
\begin{equation}
	u = \frac{1}{2} Q \Delta \varphi = \frac{1}{2} C \Delta \varphi ^2
\end{equation}
Si los conductores que forman un condensador tienen formas geométricas sencillas, la capacitancia puede obtenerse analíticamente. Y para condensadores en circuitos paralelos y en serie\footnote{En paralelo se tiene la misma diferencia de potencial entre los nodos y para los circuitos en serie se conserva la carga.}

\begin{align*}
	C_{eq} &= C_1 + C_2, \qquad \text{Paralelo} \\
	\frac{1}{C_{eq}} &= \frac{1}{C_1} + \frac{1}{C_2}, \qquad \text{Serie.}
\end{align*}



\chapter{Corriente Eléctrica}
Carga en movimiento $\to$ corriente eléctrica. Proceso por el cual se transporta la carga $\to$ condicción. La corriente se define como la velocidad a la que se transporta la carga a través de una superficie dad e un sistema conductor
\begin{equation}
	I = \dv{Q}{t}.
\end{equation}
Por convención: el sentido en que se mueve los portadores positivos se toma como el sentido de la corriente. \\

Cosas a notar:
\begin{itemize}
	\item En gases la conducción es más complicada, ya que las poblaciones d eportadores varían mucho con condiciones experimentales.
	\item Al estar en equilibrio térico cada partícula tiene un movimiento aleatorio.
	\item Líquidos y gases $\to$ movimientos hidrodinámicos.
\end{itemize}

\section{Densidad de Corriente}
Consideremos un medio conductor con un solo tipo de portador de carga:
\begin{itemize}
	\item Sea $N$ un número de portadores de carga por unidad de volumen.
	\item Asumimos una velocidad $v$ para los portadores de carga.
\end{itemize}
Se define la densidad de carga como
\begin{equation}
	\vec{J} = \sum _i N_i q_i v_i.
\end{equation}
Cuya integral para una superficie da como resultado la corriente
\begin{equation}
	I = -\oint _S \vec{J} \cdot \vu{n} \dd{a} = -\int _V \div{\vec{J}} \dd{V} = \int _V \pdv{\rho}{t} \dd{V}.
\end{equation}
Lo que implica y da como resultado la \textbf{ecuación de la continuidad} y representa la \textbf{conservación de la carga}
\begin{equation}
	\pdv{\rho}{t} + \div{\vec{J}} = 0.
\end{equation}


\section{Ley de Ohm}









\chapter{El Campo Magnético de Corrientes Estacionarias}









\chapter{Propiedades Magnéticas de la Materia}













\chapter{Inducción Electromagnética}














\chapter{Energía Magnética}













\chapter{Corrientes que Varían Lentamente}















\chapter{Ecuaciones de Maxwell}














\chapter{Propagaricón de Ondas Monocromáticas}










































%%%%%%%%%%%%%%%%%5