\part{Reducción de Datos}

\vspace*{\fill}

\begin{center}
	\textit{''La estadística es la gramática de la ciencia.'' - Karl Pearson.}
\end{center}

\vspace*{\fill}

%\chapter{Introducción al Cálculo de Probabilidades en Estadística}


\chapter{Incertezas}
\section{Uso y Reporte de Incertezas}
\begin{equation}
	\boxed{ \text{valor medido de } x = x_{\text{mejor}} \pm \delta x .}
\end{equation}
Donde $\delta x$ siempre es positivo.

\begin{tcolorbox}
	\begin{center}
		\textbf{Regla para Escribir incertezas}
	\end{center}
	Las incertezas debe, casi siempre, estar aproximadas a la cifra significativa.
\end{tcolorbox}


\begin{tcolorbox}
	\begin{center}
		\textbf{Regla para Escribir Respuestas}
	\end{center}
	La última cifra significativa en cualquier resultado debe tener el mismo orden de magnitud (en la misma posición decimal) que la incerteza.
\end{tcolorbox}

\begin{definition}
	La discrepancia esta definida como la diferencia entre 2 valores medidos de la misma cantidad.
\end{definition}

\begin{definition}
	$$\text{incerteza fraccionaria o relativa } = \frac{\delta x}{\abs{x_{\text{mejor}}}}.$$
\end{definition}




\section{Propagación de Incertezas}

\subsection{La Regla de Raíz Cuadrada para un Experimento de Conteos}
\begin{definition}
	Si observamos la ocurrencias de un evento que pasa aleatoriamente peor con un promedio definido, si se tienen $\nu$ ocurrencias en un tiempo $T$, nuestra estimación para el promedio es
	\begin{equation}
		(\text{promedio de número de eventos en el tiempo } T) = \nu \pm \sqrt{\nu}.
	\end{equation}
\end{definition}

\subsection{Reglas de Propagación de Error}
La sreglas de propagación de error se refiere a uan situación en la cual encontramos varias cantidades $x,\ldots,w$ con incertezas $\delta x, \ldots ,\delta w$ y cuando usamos estos valores para calcular $q$. 
\begin{tcolorbox}
	\begin{description}
		\item[Sumas y Restas: ] $q = x + \cdots + z - (u + \cdots + w)$, entonces
			\begin{equation}
				\delta q = \sqrt{\delta x ^2 + \cdots + \delta z ^2 + \delta u ^2 + \cdots + \delta w^2}.
			\end{equation}
		\item[Productos y Cocientes: ] Si $q = \frac{x\cdots z}{u \cdots w},$
			\begin{equation}
				\frac{\delta q}{\abs{q}} = \sqrt{\qty(\frac{\delta x}{x})^2 + \cdots + \qty(\frac{\delta z}{z})^2 + \qty(\frac{\delta u}{u})^2 + \cdots + \qty(\frac{\delta w}{w})^2}.
			\end{equation}
		\item[Incerteza de una Potencia: ] Si $n$ es un numero exacto $q = x^n$
			\begin{equation}
				\frac{\delta q}{\abs{q}} = \abs{n} \frac{\delta x}{\abs{x}}.
			\end{equation}
		\item[Incerteza de una función de una Variable: ] Si $q = q(x)$ es una función de $x$
			\begin{equation}
				\delta q = \abs{\dv{q}{x}} \delta x.
			\end{equation}
		o en caso de que $q$ sea muy complicada
			\begin{equation}
				\delta q = \abs{q(x_{best} + \delta x) - q(x_{best})}.
			\end{equation}
		\item[Fórmula General de la Propagación de Error: ] Si $q = q(x,\ldots ,z)$ es una función de $x,\ldots ,z$, entonces
			\begin{equation}
				\delta q = \sqrt{\qty(\pdv{q}{x} \delta x)^2 + \cdots + \qty(\pdv{q}{z} \delta z)^2}.
			\end{equation}
	\end{description}
\end{tcolorbox}


\section{Análisis Estadístico de Incertezas Aleatorias}
\subsection{Promedio o Media}
\begin{tcolorbox}
	\begin{description}
		\item[La Media: ] El mejor valor estimado para $x$, es la media (en este caso: aritmética)
			\begin{equation}
				\bar{x} = \frac{1}{N} \sum _{i=1} ^N x_i.
			\end{equation}
		\item[Desviación Estandar: ] El promedio de las incertezas de las mediciones individuales
			\begin{equation}
				\sigma _x = \sqrt{\frac{1}{N - 1} \sum (x_i - \bar{x})^2}.
			\end{equation}
		Anteriormente el denominador se tomaba como $N$. Además, podemos identificar a $\sigma _x$ como la incerteza de cualquier medición de $x$, $\delta x = \sigma _x$, y podemos decir con un $68\%$ de confianza que cualquier mediciónd e $x$ caerá dentro de $\sigma _x$.
		\item[Desviación Estandar de la Media: ] La incerteza de nuestro mejor valor (la media) es:
			\begin{equation}
				\sigma _{\bar{x}} = \frac{\sigma _x}{\sqrt{N}}.
			\end{equation}
		Si hay errores sistemáticos apreciables, entonces $\sigma _{\bar{x}}$ da la \textit{componente aleatoria} de la incerteza en nuestra mejor estimación de $x$
			\begin{equation}
				\delta x_{ran} = \sigma _{\bar{x}}.
			\end{equation}
		Si existe una forma para estimar la componente sistematica $\delta x_{sys}$, una rasonable (más no rigurosamente justificada) expresión para la incerteza total es la suma de los cuadrados entre ambas incertezas
			\begin{equation}
				\delta x_{tot} = \sqrt{\delta x_{ran}^2 + \delta x_{sys}^2}.
			\end{equation}
	\end{description}
\end{tcolorbox}


\chapter{La Distribución Normal}
\section{Distribución Límite}
		Si $f(x)$ es la distribución límite de una variable continua $x$, entonces
			\begin{align*}
				f(x) \dd{x} = \text{ probabilidad que cualesquiera de las mediciones caerá enre } x \text{ y } x+\dd{x}.
			\end{align*}
		y
			\begin{align*}
				\int _a ^b f(x) \dd{x} = \text{ probabilidad de que cualesquiera de las mediciones caiga entre } a \text{ y } b.
			\end{align*}
		La condición de normalización es
			\begin{equation}
				\int _{-\infty} ^\infty f(x) \dd{x} = 1,
			\end{equation}
		mientras que la media o valor esperado es
			\begin{equation}
				\bar{x} = \int _{-\infty} ^\infty xf(x) \dd{x}.
			\end{equation}
\section{Distribución Normal (o de Gauss)}
Si las mediciones de $x$ estan sujetos a muchos pero pequeños errores aleatorios pero no sistemáticos, su distribución límite sera la \textit{distribución normal}:
	\begin{equation}
		G_{X,\sigma} (x) = \frac{1}{\sigma \sqrt{2\pi}} e^{-(x - X)^2 /2\sigma ^2}, 
	\end{equation}
donde $X$ es el valor real d $x$, centro de la distribución y el valor medio después muchas mediciones. Y, $\sigma$ es el parámetro de ancho de la distribución y la desviación estándar luego de muchas mediciones. \\

La probabilidad de una única medición de caer entre $t$ desviaciones estándar de $X$ es
	\begin{equation}
		P(x \leq \abs{t\sigma}) = \frac{1}{\sqrt{2\pi}} \int _{-t} ^t e^{-z^2 /2} \dd{z}.
	\end{equation}
Esta integral es normalmente llamada la función de error. Su valor como una función de $t$ es mostrado en tablas. En partícular, para $t = 1$ la probabilidad es de $68\%$.  \\

Un valor aceptado de $x$ sera aquel que caiga a cierta cantidad de desviaciones estandar del centro de la distribución, es decir, con cierto porcentaje mayor a un umbral previamente designado.


\chapter{Rechazo de Datos y la Media Ponderada}

\section{Criterio de Chauvenet}
Si se realizan $N$ mediciones $x_1 ,\ldots ,x_N$ de una única cantidad $x$, y si una de estas mediciones ($x_{sus}$) es sospechosamente diferente de las otras, el criterio de Chauvenet da una prueba para decidir si se acepta o no este dato sospechoso. Teniendo la desviación estándar y le media de las mediciones, encontraremos el número de desviaciones estándar por las cuales $x_{sus}$ difiere $\bar{x}$,
	\begin{equation}
		t_{sus} = \frac{\abs{x_{sus} - \bar{x}}}{\sigma _x}
	\end{equation}
Luego encontramos la probabilidad (asumiendo que los valores están normalmente distribuidos alrededor de $\bar{x}$ con ancho $\sigma _x$) de encontar este resultado tan desviado como $x_{sus}$, y el número de mediciones esperadas para desviarse esta cantidad es $n = N \times P(\text{fuera } t_{sus} \sigma)$. Si $n< \frac{1}{2}$, entonces de acuerdo al criterio de Chauvenet puedes rechazar el valor de $x_{sus}$. Dado que existen muchas objeciones al criterio de Chauvenet (especialmente cuando $N$ no es muy grande), esto solo debería ser utilizado como último recurso. Además, esto se pierde \dsnote{se va al carajo} si dos o más valores son sospechosos.

\section{Media Ponderada}
Si $x_1 ,\ldots ,x_N$ son mediciones de una cantidad $x$, con incertezas conocidas $\sigma _1,\ldots ,\sigma _N$, entonces el valor mejor estimado al valor real de $x$ es la media ponderada
	\begin{equation}
		x_{wav} = \frac{\sum w_i x_i}{\sum w_i},
	\end{equation}
donde als sumas son sobre todas las mediciones y los pesos son los reciprocos cuadrados de las incertezas correspondientes
	\begin{equation}
		w_i = \frac{1}{\sigma _i ^2}.
	\end{equation}
La incerteza de $x_{wav}$ es
	\begin{equation}
		\sigma _{wav} = \frac{1}{\sqrt{\sum w_i}}.
	\end{equation}



\chapter{Distribución Binomial y de Poisson}
\section{La Distribución Binomial}
Conseideraremos un experimento con varios posibles resultados y designamos los resultado (o resultados) particular en el que estamos interesados como un ''éxito''. Si la probabilidad de éxito en cualquier ensayo es $p$, entonces la probabilidad de $\nu$ éxitos en $n$ ensayos viene dada por la distribución binomial:
	\begin{equation}
		P(\nu \text{ éxitos en } n \text{ intentos}) = \binom{n}{\nu} p^\nu (1 - p)^{n - \nu},
	\end{equation}
Si repetimos el conjunto completo de $n$ ensayos muchas veces, esperamos que el número medio de éxitos es
	\begin{equation}
		\bar{\nu} = np
	\end{equation}
y su desviación estándar es
	\begin{equation}
		\sigma _\nu = \sqrt{np(1 - p)}.
	\end{equation}

\subsection{Aproximación Gaussiana a la Distribución Binomial}
Cuando $n$ es grande, la distribución binomial $B_{n,p} (\nu)$ esta bien aproximada por la función de Gauss con la misma media y desviación estándar, esto es
	\begin{equation}
		B_{n,p} (\nu) \approx G_{X,\sigma} (\nu).
	\end{equation}


\section{Distribución de Poisson}
La distribución de Poisson describe experimentos en los cuales se cuentan eventos que ocurren aleatoriamente pero a una tasa promedio definida. SI se cuentan durante un intervalo $T$, la probabilidad de observar $\nu$ eventos es dada por la función de Poisson
	\begin{equation}
		P_\mu (\nu) = e^{-\mu} \frac{\mu ^\nu}{\nu !},
	\end{equation}
donde el parámetro $\mu$ es el numero promedio esperado de eventos en el tiempo $T$: $\bar{\nu} = \mu$. Su desviación estándar es $\sigma _\nu = \sqrt{\mu}$.

\subsection{Aproximación Gaussiana a la Distribución de Poisson}
Cuando $\mu$ es grande, la distribución de Poisson se aproxima bien a una función Gaussiana con la misma media y desviación estandar:
	\begin{equation}
		P_{\mu} (\nu) \approx G_{X,\sigma} (\nu).
	\end{equation}



\chapter{Prueba Ji Cuadrado y Mínimos Cuadrados}
\section{Mínimos Cuadrados}
Se considerarán $N$ pares de mediciones $(x_1 ,y_1),\ldots ,(x_N ,y_N)$ de dos variables. El problema recae en encontrar los mejores valores de los parámetros de la curva que una gráfica $y$ vs $x$ se espera que ajuste.

\subsection{Una Línea Recta; Ponderaciones Iguales}
Si $y$ se espera que caiga en una línea recta $y = A + Bx$, si las medidas de $y$ tienen todas la misma incerteza, las mejores aproximaciones para las constantes son
	\begin{equation}
		A = \frac{\sum x^2 \sum y - \sum x \sum xy}{\Delta}
	\end{equation}
y
	\begin{equation}
		B = \frac{N\sum xy - \sum x \sum y}{\Delta},
	\end{equation}
donde $\Delta$ es
	\begin{equation}
		\Delta = N\sum x^2 - \qty(\sum x)^2.
	\end{equation}
Basados en las observaciones, la mejor estimación para las incertezas son
	\begin{align}
		\sigma _y &= \sqrt{\frac{1}{N - 2} \sum _{i=1} ^N (y_i - A - Bx_i)^2}, \\
		\sigma _A &= \sigma _y \sqrt{\frac{\sum x^2}{\Delta}}, \\
		\sigma _B &= \sigma _y \sqrt{\frac{N}{\Delta}}.
	\end{align}

\subsection{Línea Recta por el Origen; Ponderaciones Iguales}
Si $y$ se espera que caiga en una línea recta que atraviese el origen $y = Bx$, si las mediciones de $y$ todas tienen la misma incerteza, la mejor estimación para las constantes es
	\begin{equation}
		B = \frac{\sum xy}{\sum x^2}.
	\end{equation}
Basados en las observaciones, las incertezas son
	\begin{align}
		\sigma _y &= \sqrt{\frac{1}{N - 1} \sum (y_i - Bx_i)^2}, \\
		\sigma _B &= \frac{\sigma _y}{\sqrt{\sum x^2}}.
	\end{align}

\subsection{Ajuste Ponderado para una Línea Recta}
Si $y$ se espera que sea una línea recta, y los valores medidos de $y$ tienen diferentes y conocidas incertezas $\sigma _i$, introducimos las ponderaciones $w_i = \frac{1}{\sigma _i ^2}$, las mejores estimaciones de las constantes son
	\begin{align}
		A &= \frac{\sum wx^2 \sum wy - \sum wx \sum wxy}{\Delta}, \\
		B &= \frac{\sum w \sum wxy - \sum wx \sum wy}{\Delta}, \\
		\Delta &= \sum w \sum wx^2 - \qty(wx)^2.
	\end{align}

Cuyas incertezas son
	\begin{align}
		\sigma _A &= \sqrt{\frac{\sum wx^2}{\Delta}}, \\
		\sigma _B &= \sqrt{\frac{\sum w}{\Delta}}.
	\end{align}

\subsection{Otras Curvas}
Si $y$ se supone como un polinomio $y = A + Bx + \cdots + Hx^n$, entonces existe un método análogo, pero las ecuaciones son bastante engorrosas. También curvas de la forma $y = A f(x) + \cdots + Hk(x)$, donde las funciones son conocidas, existe un método análogo también. Otra forma es linearizar el problema; por ejemplo, para una función exponencial, su ''linearización'' es $z = \ln{y} = \ln{A} + Bx$.




\section{Prueba Chi-Cuadrado para Distribuciones}
\subsection{Definicicón de Chi-Cuadrado}
Si se realizan $n$ mediciones para las cuales se sabe, o se puede calcular, los valores esperados y las desviaciones estándar, entonces, definimos $\chi ^2$ como
	\begin{equation}
		\chi ^2 = \sum _i ^n \qty(\frac{\text{Valor observado } - \text{ valor esperado}}{\text{desviación estándar}})^2.
	\end{equation}
Las $n$ mediciones son los números, $O_1 ,\ldots ,O_n$, de veces que el valor de una cantidad $x$ fue observado en cada una de las $n$ ''bins''. En este caso, el número esperado $E_k$ es determinado por la distribución propuesta de $x$, y la desviación estándar es $\sqrt{E_k}$; por lo tanto
	\begin{equation}
		\chi ^2 = \sum _{k=1} ^n \frac{(O_k - E_k)^2}{E_k}.
	\end{equation}
\begin{tcolorbox}
	Si la distribución propuesta para $x$ es correcta, entonces $\chi ^2$ debería ser de orden $n$. Si $\chi ^2 \gg n$, la distribución propuesta es probablemente incorrecta.
\end{tcolorbox}

\subsection{Grados de Libertad y Chi-Cuadrado Reducido}
Si se repetirá el momento muchas veces, el valor medio de $\chi ^2$ debería ser igual a $d$, el número de \textit{grados de libertad}, definido como
	\begin{equation}
		d = n - c,
	\end{equation}
donde $c$ es el número de \textit{restricciones}, el número de parámetros que se han tenido que calcular desde los datos para encontrar $\chi ^2$. Chi-Cuadrado reducido se define como
	\begin{equation}
		\overset{\sim}{\chi} ^2 = \chi ^2 /d.
	\end{equation}
Si la distribución propuesta es correcta, $\overset{\sim}{\chi} ^2$ debería de ser de orden $1$; si $\overset{\sim}{\chi} ^2 \gg 1$, los datos no se ajustan a la distribución propuesta satisfactoriamente.

\subsection{Probabilidades para Chi-Cuadrado}
Suponga que obtiene $\overset{\sim}{\chi} _o ^2$ de un experimento. Si $\overset{\sim}{\chi} _o ^2$ es apreciablemente mayor a $1$, se tienen razones para dudar de la distribución sobre la cual sus valores esperados $E_k$ están basados. Desde una tabla de probabilidades se puede encontrar
	\begin{equation}
		P(\overset{\sim}{\chi} ^2 \geq \overset{\sim}{\chi} _o ^2),
	\end{equation}
probabilidad de obtener un valor $\overset{\sim}{\chi} ^2$ tan grande como $\overset{\sim}{\chi} _o ^2$, asumiendo que la distribución esperada es la correcta. Si esta probabilidad es pequeña, tiene motivos para rechazar la distribución esperada; si es inferior al $5\%$, se rechazaría la distribución supuesta al nivel del $5\%$, o nivel significativo; si la probabilidad es inferior al $1\%$, se rechazaría la distribución al nivel del $1\%$, o nivel altamente significativo.
























%%%%%%%%%%