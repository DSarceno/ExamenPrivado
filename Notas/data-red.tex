\part{Reducción de Datos}

\vspace*{\fill}

\begin{center}
	\textit{''La estadística es la gramática de la ciencia.'' - karl Pearson.}
\end{center}

\vspace*{\fill}

%\chapter{Introducción al Cálculo de Probabilidades en Estadística}


\chapter{Incertezas}
\section{Uso y Reporte de Incertezas}
\begin{equation}
	\boxed{ \text{valor medido de } x = x_{\text{mejor}} \pm \delta x .}
\end{equation}
Donde $\delta x$ siempre es positivo.

\begin{tcolorbox}
	\begin{center}
		\textbf{Regla para Escribir incertezas}
	\end{center}
	Las incertezas debe, casi siempre, estar aproximadas a la cifra significativa.
\end{tcolorbox}


\begin{tcolorbox}
	\begin{center}
		\textbf{Regla para Escribir Respuestas}
	\end{center}
	La última cifra significativa en cualquier resultado debe tener el mismo orden de magnitud (en la misma posición decimal) que la incerteza.
\end{tcolorbox}

\begin{definition}
	La discrepancia esta definida como la diferencia entre 2 valores medidos de la misma cantidad.
\end{definition}

\begin{definition}
	$$\text{incerteza fraccionaria o relativa } = \frac{\delta x}{\abs{x_{\text{mejor}}}}.$$
\end{definition}




\section{Propagación de Incertezas}

\subsection{La Regla de Raíz Cuadrada para un Experimento de Conteos}
\begin{definition}
	Si observamos la ocurrencias de un evento que pasa aleatoriamente peor con un promedio definido, si se tienen $\nu$ ocurrencias en un tiempo $T$, nuestra estimación para el promedio es
	\begin{equation}
		(\text{promedio de número de eventos en el tiempo } T) = \nu \pm \sqrt{\nu}.
	\end{equation}
\end{definition}

\subsection{Reglas de Propagación de Error}
La sreglas de propagación de error se refiere a uan situación en la cual encontramos varias cantidades $x,\ldots,w$ con incertezas $\delta x, \ldots ,\delta w$ y cuando usamos estos valores para calcular $q$. 
\begin{tcolorbox}
	\begin{description}
		\item[Sumas y Restas: ] $q = x + \cdots + z - (u + \cdots + w)$, entonces
			\begin{equation}
				\delta q = \sqrt{\delta x ^2 + \cdots + \delta z ^2 + \delta u ^2 + \cdots + \delta w^2}.
			\end{equation}
		\item[Productos y Cocientes: ] Si $q = \frac{x\cdots z}{u \cdots w},$
			\begin{equation}
				\frac{\delta q}{\abs{q}} = \sqrt{\qty(\frac{\delta x}{x})^2 + \cdots + \qty(\frac{\delta z}{z})^2 + \qty(\frac{\delta u}{u})^2 + \cdots + \qty(\frac{\delta w}{w})^2}.
			\end{equation}
		\item[Incerteza de una Potencia: ] Si $n$ es un numero exacto $q = x^n$
			\begin{equation}
				\frac{\delta q}{\abs{q}} = \abs{n} \frac{\delta x}{\abs{x}}.
			\end{equation}
		\item[Incerteza de una función de una Variable: ] Si $q = q(x)$ es una función de $x$
			\begin{equation}
				\delta x = \abs{\dv{q}{x}} \delta x.
			\end{equation}
		o en caso de que $q$ sea muy complicada
			\begin{equation}
				\delta q = \abs{q(x_{best} + \delta x) - q(x_{best})}.
			\end{equation}
		\item[Fórmula General de la Propagación de Error: ] Si $q = q(x,\ldots ,z)$ es una función de $x,\ldots ,z$, entonces
			\begin{equation}
				\delta q = \sqrt{\qty(\pdv{q}{x} \delta x)^2 + \cdots + \pdv{q}{z} \delta z)^2}.
			\end{equation}
	\end{description}
\end{tcolorbox}







\section{Análisis Estadístico de Incertezas Aleatorias}





\chapter{La Distribución Normal}










\chapter{Rechazo de Datos y la Media Ponderada}











\chapter{Distribución Binomial y de Poisson}










\chapter{Prueba Ji Cuadrado y Mínimos Cuadrados}













































%%%%%%%%%%