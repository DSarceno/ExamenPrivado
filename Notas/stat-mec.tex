\part{Mecánica Estadística}

\chapter{Entropía y Temperatura}

\section{Macroestados y Microestados}

Un \textbf{microestado} es la especificación detallada de una configuración microscópica de un sistema termodinámico. En otras palabras, un microestado es un punto del espacio fásico de dicho sistema. Mientras que un \textbf{macroestado} se refiere a una caracterización de un sistema termodinámico mediante los valores de un número finito de $n$ variables de estado, de las cuales al menos una debe ser extensiva. Un macroestado viene dado por una distribución de probabilidad sobre un conjunto dado de microestados; en función del conjnto de microestados considerando, la distribución toma una u otra forma. Un sistema en equilibrio permanece en un macroestado (macroestado de equilibrio) mientras visita los diferentes microestados accesibles a lo largo de sus fluctuaciones.