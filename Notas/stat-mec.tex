\part{Mecánica Estadística}


\vspace*{\fill}

\begin{center}
	\textit{Ludwig Boltzmann, quien dedicó gran parte de su vida a estudiar Mecánica Estadística, murió en 1906, por su propia mano. Paul Ehrenfest, el cual continuó con su trabajo, murió de manera similar en 1933. Ahora nos toca a nosotros... \\
    El plan de la Mecánica Estadística es establecer una conexión entre el nivel microscópico descrito por la mecánica y esos mismos sistemas considerados pero a nivel macroscópico.}
\end{center}

\vspace*{\fill}




\chapter{Entropía y Temperatura}

\section{Macroestados y Microestados}

Un \textbf{microestado} es la especificación detallada de una configuración microscópica de un sistema termodinámico. En otras palabras, un microestado es un punto del espacio fásico de dicho sistema. Mientras que un \textbf{macroestado} se refiere a una caracterización de un sistema termodinámico mediante los valores de un número finito de $n$ variables de estado, de las cuales al menos una debe ser extensiva. Un macroestado viene dado por una distribución de probabilidad sobre un conjunto dado de microestados; en función del conjnto de microestados considerando, la distribución toma una u otra forma. Un sistema en equilibrio permanece en un macroestado (macroestado de equilibrio) mientras visita los diferentes microestados accesibles a lo largo de sus fluctuaciones.


\section{Ensambles}

Un ensabmle estadístico (colectividad estadística) se define como un conjunto hipotético de sistemas termodinámicos de características similares que nos permiten realizar un análisis estadístico de dicho conjunto, en otras palabras, un conjunto de microestados. Existen varios tipos de ensambles:
\begin{description}
    \item[Ensamble Microcanónico: ] Un ensamble de sistemas termodinámicos que no intercambian energía ni materia con el entorno.
    \item[Ensamble Canónico: ] Un ensamble de sistemas que intercambian energía pero no materia con el entorno.
    \item[Ensamble Macrocanónico: ] Un ensamble de sistemas que intercambian materia y energía con el ambiente.
\end{description}

La forma de función de partición para cada tipo de ensamble es:
\begin{description}
    \item[Microcanónico: ] $\Omega (U,V,N) = e^{\beta TS}$, sistema cerrado y aislado (energía constante y entropía máxima).
    \item[Canónico: ] $Z(T,V,N) = e^{-\beta A}$, sistema cerrado con energía variable y temperatura fijada.
    \item[Macrocanónico: ] $\Xi (T,V,\mu) = e^{\beta pV}$\footnote{Donde $\mu$ es el potencial químico.}, sistema abierto.
\end{description}


\section{Conteos}
Técnicas básicas de conteo y sus fórmulas. Estas serán importantes para la deducción de las estadísticas o distribuiones de Boltzmann, Fermi-Dirac y Bose-Einstein.
\subsection{Conteos Básicos}

\begin{description}
    \item[Cardinalidad: ] Sea $A$ un conjunto finito, la cardinalidad de $A$ ($\abs{A}$) es el número de elementos de $A$. 
    \item[Conjuntos Distintos: ] Dos conjutnos $A$ y $B$ son distintos ssi $A\cap B = \varnothing$.
    \item[Regla de la Suma: ] Sean $A$ y $B$ conjuntos distintos $\abs{A\cup b} = \abs{A} + \abs{B}$, esto es válido para $n$ conjuntos distintos.
    \item[Producto Cartesiano: ] Sea $A$ y $B$ dos conjuntos cualesquiera, el producto cartesiano $A \times  B$ se define de la siguiente forma
        $$ A \times B = \{ (a,b) \, | \, a\in A,\, b\in B \} . $$
        Igual que la anterior, esto es válido para $n$ conjuntos cualesquiera.
    \item[Regla de la Multiplicación: ] $\abs{A_1 \times \cdots \times A_n} = \abs{A_1} \cdots \abs{A_n}$.
\end{description}

Casos de conteo básico
\begin{description}
    \item[Disposiciones: ]  Sea $A$ un conjunto con $n$ elementos. Una disposición de rango $k$ del conjunto $A$ es una elección (escogencia) de $k$ elementos de $A$ donde:
    \begin{enumerate}
        \item Si se puede repetir
        \item Si importa el orden
    \end{enumerate}
    $D_n ^k = $ Conjunto de disposiciones de $k$ elemento del conjunto $A$.
        $$  \boxed{ \abs{D_n ^k} = n^k . } $$
    \item[Permutaciones: ] Sea $A$ un conjunto con $n$ elementos. Una permutación de rango $k\leq n$ es una elección de $k$ elementos de $A$ donde: 
    \begin{enumerate}
        \item No se puede repetir
        \item Si importa el orden
    \end{enumerate}
    $\mathcal{P}_n ^k = $ Conjunto de permutaciones. $P_n ^k = \abs{\mathcal{P}_n ^k} = $ Número de permutaciones.
        $$ \boxed{ P_n ^k = \frac{n!}{(n - k)!}. } $$
    \item[Ordenaciones: ] Una ordenación es un caso especial de permutaciones, donde se eligen los $n$ elementos del conjutno $A$. Osea que una ordenación es una permutación donde $k=n$.
        $$ \boxed{ \text{Número de Ordenaciones} = n!. } $$
    \item[Permutaciones con Repetición (Boltzmann): ] Sea $A$ un conjunto con $n$ elementos y vamos a escoger $k$ elementos donde sí importa el orden y el elemento $a_i$ se repite $k_i$ veces. A este tipo de escogencia se le llama permutación con repetición.
        $$ \boxed{ \text{Número de Permutaciones con Repetición} = \frac{k!}{k_1 ! \cdots k_n !}. } $$
    Debido a que $a_i$ lo escogemos $k_i$ veces y si diferenciamos cada elección de $a_i$ formaríamos un conjunto con $k$ elementos y estos $k$ elementos se pueden ordenar de $k!$ formas, pero luego no lo diferenciamos y tendríamos $k_i$ ordenaciones iguales y por lo tanto dividimos por $k_i!$ para todo $i$ para contar las ordenaciones diferentes.
\end{description}

\begin{tcolorbox}
El ensamble microcanónico es el conjunto de todos los microestados que tienen la distribución permitida de máxima entropía.
\end{tcolorbox}

\begin{description}
    \item[Coeficiente Binomial: ] 
        $$ \mqty(n \\ k) = \frac{n!}{k! (n - k)!}. $$
    \item[Propiedad 1: ] Simetría
        $$ \mqty(n \\ k) = \mqty(n \\ n - k). $$
    \item[Propiedad 2: ] Triángulo de Pascal
        $$ \mqty(n \\ k) + \mqty(n \\ k + 1) = \mqty(n + 1 \\ k + 1). $$
    \item[Binomio de Newton: ] 
        $$ (x + y)^n \sum _{k=0} ^n \mqty(n \\ k) x^{n - k} y^k . $$
    \item[Teorema: ] 
        $$ \sum _{k=0}  ^n \mqty(n \\ k) = 2^n . $$
    \item[Combinaciones (Fermi-Dirac): ] Sea $A$ un conjunto con $n$ elementos. Una combinación de $k$ elementos en $n$ elementos es una elección de $k$ elementos del conjunto $A$ donde
    \begin{enumerate}
        \item No se puede repetir
        \item No importa el orden
    \end{enumerate}
    $\mathcal{C} _k ^n = \{ \text{Combinaciones de k elementos en n elementos}. \}$ Priemro elijamos $k$ elementos en forma ordenada, como si fueran permutaciones y luego dividimos entre todas las ordenaciones de los $k$ elementos.
        $$ \boxed{ C_k ^n = \mqty(n \\ k). } $$
    \item[Distribución (Bose-Einstein): ] Sea $A$ un conjunto de $n$ elementos. Una distribución es una elección de $k$ elementos de $A$ donde:
    \begin{enumerate}
        \item Si se puede repetir
        \item No importa el orden
    \end{enumerate}
    $\mathcal{D} _k ^n = $ Distribuciones de $k$ en $n$.
        $$ \text{Número de Distribuciones} = \mqty(n - 1 + k \\ k) = \mqty(n - 1 + k \\ n - 1). $$
\end{description}

\subsection{Fórmula de Stirling}
La fórmula de Stirling es una aproximación de la función factorial de un número natural $n$, que es especialmente útil para grandes valores de $n$.
    $$ n! \approx \sqrt{2\pi n} \qty(\frac{n}{e})^n, $$
esta aproximación puede representarse también de forma logarítmica
    $$ \ln{n!} \approx n\ln{n} - n + \frac{1}{2} \ln{2\pi n}. $$

La precisión de esta fórmula mejora a medida que $n$ aumenta.


\section{Entropía y Función de Partición}
En nuestro problema básico de Mecánica Estadística tenemos $n$ partículas distinguibles entre sí y tenemos $k$ estados y en cada estado pueden haber cualquier número de partículas. Además cada estado se identifica con su nivel de energía. Diferentes estados pueden tener el mismo nivel de energía. Lo anterior lo decimos, formalmente, que la energía puede estar degenerada. En general, la energía no nos sirve de índice; la energía sirve de índice solamente cuando no hay degeneración. Siempre es requerido conocer la función de degeneración. \\

En este problema tenemos 2 restricciones, el número de partículas es $n$ y la energía total es $E$. Lo único que se respeta son esass dos restricciones. Las partículas solamente obedecen la srestricciones, todo lo demás es completamente aleatorio. Cuando una distribución respeta las restriccioens decimos que es una distribución admisible o posible; cuando una distribución no respeta las restricciones decimos que es una distribución imposble o inadmisible. En este momento una distribución es función que le asigna $n_i$ partículas al estado $E_i$; osea que la función va sobre los índices. \\

Osea que, una distribución se puede escribir de a siguiente forma
	$$ (n_1 ,\ldots ,n_k) $$
y está sujeta a las siguientes restricciones
	$$ \sum _{i=1} ^k n_i = n \qquad \sum _{i=1} ^k n_i E_i = E. $$

También tenemos que, por ahora, las partículas son distinguibles por lo tanto definimos como microestado a una función que asigna a acada partícula un estado. Los micrestados que dan una distribución admisible se llaman microestados admisibles o posibles. Los microestados que dan una distribución inadmisible o imposible se llaman microestados inadmisibles o imposible. \textit{Microestados diferentes pueden dar la misma distribución.}

\subsection{Postulado Básico}
\begin{tcolorbox}
	Todos los microestados admisibles tienen la misma probabilidad de salir.
\end{tcolorbox}


\begin{tcolorbox}
	Microestados inadmisibles tienen probabilidad cero de salir, son imposibles.
\end{tcolorbox}

Como consecuencia de lso postulados, la distribución más probable es la distribución que tenga más microestados admisibles. Por lo tanto, tenemos que contar microestados de cada distribución admisible y luego escoger la que tenga más microestados admisibles.

\subsection{Conteo de Microestados}
El número de microestados admisibles de la distribución admisible $(n_1 ,\ldots ,n_k)$ con $n = n_1 + \cdots + n_k$ es:
	$$ \Omega (n_1 ,\ldots ,n_k) = \text{Número de microestados de la distribución.} $$

	$$ \boxed{ \Omega (n_1 ,\ldots ,n_k) = \frac{n!}{n_1 ! \cdots n_k !}. } $$
\textbf{\textit{QUEREMOS MAXIMIZAR}} $\mathbf{\mathit{\Omega}}$!


\subsection{Problema Básico de Mecánica Estadística}
Maximizar
	$$ \Omega (n_1 ,\ldots ,n_k) = \frac{n!}{n_1 ! \cdots n_k !} $$
sujeto a 
	$$ \sum _{i=1} ^k n_i = n \qquad \sum _{i=1} ^k n_i E_i = E. $$
Para facilitar la solución se utiliza la fórmula de Stirling. Con esto llegamos a que $p_i = \frac{n_i}{n}$ cuya interpretación es probabilidad. De lo anterior tenemos que
	$$ \sum _{i=0} ^k p_i = 1. $$
\textbf{Teoría de probabilidades}, es el estudio de las variables aleatorias y sus propiedades. \textbf{Estadística}; es el estudio y desarrollo de teorías y técnicas para medir, establecer, calcular o estimar, variables aleatorias. Continuando con el procedimiento de maximizar, se tiene que
	$$ \boxed{ S = -k_B \sum _{i=0} ^k p_i \log{p_i}. } $$
	
Ahora tenemos la siguiente equivalencia de dos problemas
\begin{tcolorbox}
	Maximizar
	$$ \Omega (n_1 ,\ldots ,n_k) = \frac{n!}{n_1 ! \cdots n_k !} $$
sujeto a 
	$$ \sum _{i=1} ^k n_i = n \qquad \sum _{i=1} ^k n_i E_i = E. $$
\end{tcolorbox}
$\Leftrightarrow$
\begin{tcolorbox}
	Maximizar
	$$ S = -k_B \sum _{i=0} ^k p_i \log{p_i} $$
sujeto a 
	$$ \sum _{i=1} ^k p_i = 1 \qquad \sum _{i=1} ^k p_i E_i = E. $$
\end{tcolorbox}

Maximizando la segunda equivalencia se llega a que
	$$ \mathcal{z} = e^{1 + \alpha} \qquad \alpha + 1 = \log{\mathcal{z}} $$
entonces 
	$$ \boxed{ \mathcal{z} (\beta) = \sum _{i=1} ^k e^{-\beta E_i}, \quad \beta = \frac{1}{k_B T}. } $$


\subsection{Valor Esperado de $E_i$}
Ahora calculamos el valor esperado de la variable aleatoria $E_i$.
    $$ \expval{E_i} = \sum _{i=0} ^k p_i E_i . $$
La forma de calcular o estimar una variable aleatoria es muestreandola. \\

Para la distribución de Boltzmann, podemos calcular el valor esperado de la siguiente forma, usando la función de partición.

    $$ \partition = \sum _{i=0} ^k e^{-\beta E_i},  $$
derivando respecto a $\beta$ se tiene que
    $$ \frac{1}{\partition} \dv{\partition}{\beta} = -\varepsilon . $$
A continuación vamos a ver que $\log{\partition}$ juega un papel importante en Mecánica Estadística. Derivando $\log{\partition}$ tenemos lo siguiente usando la regla de la cadena.
    $$ \dv{\log{\partition}}{\beta} = -\varepsilon . $$


\subsection{Entropia v2}
La entropía como se mostró anteriormente es como una densidad de entropía. Ahora, operando llegamos a que
    $$ S =  k_B \qty(\beta \varepsilon + \log{\partition}). $$
y ojo que $S$ no depende de $\beta$ (esto se puede probar diferenciando la expresión anterior llegamos a que $\pdv{S}{\beta} = 0$). De esto tenemos que
    $$ \boxed{ \dd{S} = \frac{\dd{\varepsilon}}{T} \qquad \qquad \dv{S}{\varepsilon} = k_B \beta . } $$

\subsection{Función de Helmholtz}
    $$ F = - \frac{\log{\partition}}{\beta} $$
Tomando la definición de entropía y reemplazando $\frac{1}{T} = k_B \beta$. Entonces, se tiene
    $$ F = \varepsilon - TS. $$
Y queda claro también que $\varepsilon$ no depende solamente de $T$ sino que  también de la entropía. Notamos lo siguiente $\dd{F} = \dd{\varepsilon} - T\dd{S} - S\dd{T}$ por ende
    $$ \dd{F} = -S\dd{T} $$
Lo que implica que la función de Helmholtz depende solamente de la temperatura y no de la energía\footnote{$\varepsilon$: Energía Media} ni entropía.

\subsection{Calor Específico}
\subsubsection{Varianza}
Es conocida la definición de $VAR = \expval{E_i ^2} - \varepsilon ^2$. Esto se relaciona con la función de partición de la siguiente forma
    $$ \frac{1}{\partition} \dv[2]{\partition}{\beta} = \expval{E_i ^2}, $$
Por ende
    $$ \boxed{ \dv[2]{\log{\partition}}{\beta} = VAR (E_i). } $$
Utilizando la regla de la cadena y la definición de energía media
    $$ c_v = \dv{\varepsilon}{T}, $$
\textbf{Calor específico a volumen constante.} Por la misma regla de la cadena, se tiene que
    $$ VAR (E_i) = c_v k_B T^2. $$



\chapter{Elementos de la Teoría de los Ensambles}
\section{Generalidades de la Teoría de Ensambles}
\begin{itemize}
    \item Un microestado de un sistema clásico, en un tiempo $t$, está definido por las posiciones y momenta de todas las partículas que constituyen al sistema.
    \item Las coordenadas $(q_i ,p_i)$ representan un punto en un espacio de $6N$ dimensiones conocido como espacio de fases.
    \item Función de densidad $\rho (q,p;t)$: para describir mejor los ensambles de microestados en los que puede encontrar un sistema. Esta función es tal que el número de puntos representativos dentro del elemento de volumen $d^{3N} qd^{3N}p$ alrededor del punto $(q,p)$ del espacio de fases está dado por el producto $\rho (q,p;t) d^{3N} qd^{3N}p$.
    \item El promedio del ensamble $\expval{f}$ de una cantidad física $f(q,p)$ está dado por
        $$ \expval{f} = \frac{\int f(q,p) \rho (q,p;t) d^{3N} qd^{3N}p}{\int \rho (q,p;t) d^{3N} qd^{3N}p} . $$
    \item \textbf{Teorema de Liouville:} Consideremos una región de volumen arbitrario $\omega$, cuya superficie la vamos a denotar por $\sigma$. Entonces, la tasa a la que el número de puntos representativos en este elemento de volumen aumenta con el tiempo es
        $$ \pdv{t} \int _\omega \rho \dd{\omega}. $$
    Por otro lado, el flujo hacia afuera de $\omega$ está dado por
        $$ \int _\sigma \rho \vb{v} \cdot \vu{n} \dd{\sigma}. $$
    Por el teorema de la divergencia\footnote{$\iint _{\partial U} \vb{F} \cdot \dd{\vb{S}} = \iiint _U \div{\vb{F}} \dd{V}$, donde $S = \partial U$.}
        $$ \int _\omega \div{\rho \vb{v}} \dd{\omega}. $$
    En vista que no hay fuentes ni sumideros
        $$ \dv{t} \int _\omega \rho \dd{\omega} = -\int _\omega \div{\rho \vb{v}} \dd{\omega}, $$
    por lo que
        $$ \int _\omega \qty(\pdv{\rho}{t} + \div{\rho \vb{v}}) \dd{\omega} = 0. $$
    Por lo cual se tiene que
        $$ \pdv{\rho}{t} + \div{\rho \vb{v}} = 0, $$
    y esta ecuación es conocida como la ecuación de la continuidad. Trabajando más esta ecuación
        $$ \pdv{\rho}{t} + \sum _{i=0} ^{3N} \qty(\pdv{\rho}{q_i} \dot{q}_i + \pdv{\rho}{p_i} \dot{p}_i) + \rho \sum _{i=0} ^{3N} \qty(\pdv{\dot{q}_i}{q_i} + \pdv{\dot{p}_i}{p_i}) = 0. $$
    Recordando las ecuaciones de Hamilton:
        $$ \dot{q} _i = \pdv{H(q_i ,p_i)}{p_i}, $$
        $$ \dot{p}_i = -\pdv{H(q_i ,p_i)}{q_i}. $$
    Usando las ecuaciones de Hamilton notamos que el tercer término de la ecuación de continuidad se hace cero, por consiguiente llegamos al resultado conocido como el \textbf{teorema de Liouville}:
        $$ \pdv{\rho}{t} + \{ \rho ,H \} = 0, $$
    donde $\{ \rho ,H \}$ es el bracket de Poisson. La consecuencia física de este teorema es que las trayectorias en el espacio de fases se mueven de la misma manera que un fluido incompresible.
    \item \textbf{Ensamble Canónico: } $E = $ cte.
    \item \textbf{Ensamble microcanónico: } El macroestado del ensamble microcanónico de un sistema está definido por el número de moléculas $N$, el volumen $V$ y la energía $E$. El ensamble microcanónico es una colección de sistemas para los cuales la función de densidad $\rho$ está dada por
        $$ \rho (q.p) = cte. \qquad \qquad \text{si } \qty(E - \frac{1}{2} \Delta) \leq H(q,p) \leq \qty(E + \frac{1}{2} \Delta). $$
    \item El resultado fundamental es llegar a la energía libre de Helmholtz.
    \item El formalismo del ensamble microcanónico y canónico son equivalentes.
    \item \textbf{Teorema de Equipartición: } Cada término armónico en el Hamiltoniano transforamdo de un sistema contribuye $\frac{1}{2} kT$ a la energía interna del sistema. Dicho de otro modo, cada grado de libertad aporta la misma cantidad al valor esperado de la energía del sistema total. No obstante, el teorema de equipartición es válido para valores de temperatura muy altos, osea cuando los grados de libertado relevantes del sistema pueden ser excitados libremente.
    \item 
        $$ - \expval{\sum _i q_i \dot{p}_i} = 3NkT, $$
    donde
        $$ \mathcal{V} = -3NkT, $$
    es llamado el \" virial \" del sistema. Cuando se considera a un gas ideal esto se reduce a la relación clásica:
        $$ \mathcal{V} = -2K, $$
    con $K$ la energía cinética del sistema.
\end{itemize}

\section{Osciladores Armónicos}
Asumiendo osciladores armónicos en una dimensión el hamiltoniano $H$ del sistema es
    $$ H(q_i ,p_i) = \sum _i \frac{1}{2} m\omega ^2 q_i ^2 + \frac{1}{2m} p_i ^2 . $$
Al calcular la función de partición $\partition$ de un oscilador armónico 
    $$ \partition = \int _{-\infty} ^\infty \int _{-\infty} ^\infty \exp{-\beta \qty(\frac{1}{2} m\omega ^2 q^2 + \frac{1}{2m} p^2)} \frac{\dd{q} \dd{p}}{h}, $$
    $$ \frac{1}{h} \qty(\frac{2\pi}{\beta m\omega ^2})^{1/2} \qty(\frac{2\pi m}{\beta})^{1/2} = \frac{1}{\beta \hbar \omega} = \frac{kT}{\hbar \omega}. $$
De manera que entonces la función de partición del sistema completo es
    $$ \partition \qty(\frac{kT}{\hbar \omega})^N. $$
La energía libre de Helmholtz está dada por
    $$ A = -kT\ln{\partition} = -NkT \ln{\partition}. $$
De manera que las otras variables termodinámicas son
    $$ S = \qty(\pdv{S}{T})_{N,V} $$
    $$ = Nk\qty[\ln{\frac{kT}{\hbar \omega}} + 1] $$
y
    $$ U = \pdv{\ln{\partition}}{\beta} = NkT. $$
    
    
    
\chapter{Gas Ideal}

El gas ideal es el primer ejemplo para ilustrar la teoría que hemos desarrollado. En este ejemplo consideramos $N$ moles de átomos de un gas ideal, como el helio, en un volumen cúbico $V$ que tiene de lato $L$, osea que $V = L^3$. El cubo está aislado y esta a una presión y temperatura fija. \\
En primera instancia vamos a conceptualizar en forma clásica. Las partículas son iguales, tienen masa $m$ pero son distinguibles. ¿Qué usamos de índice para indicar los estados?
	$$ i = (x,y,z,p_x,p_y,p_z) = (\vec{r},\vec{p}) $$
¿Cuánto vale $E_i$? Vamos a ignorar la energía potencial gravitacional y otras energías potenciales; entonces $E_i$ es energía cinética
	$$ E(x,y,z,p_x,p_y,p_z) = = \frac{p^2}{2m} . $$
Ahora calculamos la función de partición (la cual es adimensional)
	$$ \partition (\beta) =  \frac{1}{h^3} \int _0 ^L \int _0 ^L \int _0 ^L \int _{-\infty} ^\infty \int _{-\infty} ^\infty \int _{-\infty} ^\infty e^{-\frac{\beta p^2}{2m}} \dd{x} \dd{y} \dd{z} \dd{p_x} \dd{p_y} \dd{p_z}. $$
Realizamos la integral y se tiene que
	$$ \partition (\beta) = \frac{L^3}{h^3} \qty(\int _{-\infty} ^\infty e^{-\frac{\beta p_x ^2}{2m}} \dd{p_x})^3. $$
Utilizando las propiedades de la función Gamma\footnote{$\int _0 ^\infty t^n e^{-at^k} \dd{t} = \frac{\Gamma \qty(\frac{(n+1)}{k})}{ka^{\frac{n+1}{k}}}$.}
    $$ \boxed{ \partition (\beta) = \frac{L^3}{h^3} \qty(\frac{2m\pi}{\beta})^{\flatfrac{3}{2}}. } $$
aplicando el logaritmo y derivando respecto a $\beta$ se tiene
    $$ \boxed{\varepsilon = \frac{3}{2} k_B T .} $$
Para el caso cuántico cambia la $h$ por $\hbar$. \\

\subsubsection{Concentración Cuántica}
$n_Q = $ Concentración cuántica
    $$ n_Q = \frac{\partition (\beta)}{V}. $$
Dado que la función de partición es adimencional; por lo tanto, las dimensionales de $n_Q$ son $\qty[1/m^3]$. Y para el gas ideal nos queda de la siguiente forma
    $$ n_Q = \frac{1}{\hbar ^3} \qty(\frac{mk_B T}{2\pi})^3 $$
\subsubsection{Longitud de Onda Térmica}
$\lambda _Q = $ Longitud de onda térmica.
    $$ \lambda _Q = n_Q ^{-\flatfrac{1}{3}} $$
Para el gas ideal nos queda
    $$ \lambda _Q = \hbar \sqrt{\frac{2\pi}{mk_B T}}. $$


\section{Cálculo de Probabilidades}

Recordemos que las probadilidades vienen dadas por
    $$ p_i = \frac{e^{-\beta E_i}}{\partition} $$
en este caso $i = (p_x,p_y,p_z) = \vec{p}$ que nos sirve para identificar los estados. Cambiaremos $i$ por $\vec{p}$ y de $p_i$ a $f(\vec{p})$. $f(\vec{p})$ es la función densidad de probabilidad sustituyendo $\partition (\beta)$
    $$ \boxed{ f(\vec{p}) = \frac{\hbar ^3}{V} \qty(\frac{2\pi \beta}{m})^{\flatfrac{3}{2}} e^{-\beta \frac{p^2}{2m}}. } $$
\subsection{Función de Densidad de $p$}
Utilizando la función densidad de probabilidad e integrando para las tres coordenadas de $p$ (en esféricas). Con lo que llegamos a 
    $$ \boxed{ g(p) = \sqrt{\frac{2}{\pi}} \qty()^{\flatfrac{3}{2}} p^2 e^{-\frac{\beta p^2}{2m}}. } $$
$g(p) = $ función de densidad de probabilidad de variable $p = \sqrt{p_x ^2 + p_y ^2 + p_z ^2}.$
\begin{description}
    \item[Moda: ] $p = \sqrt{\frac{2m}{\beta}}$.
    \item[Valor Esperado: ] $\expval{p} = \frac{2m^2}{\beta ^2} = 2k_B ^2 m^2 T^2$.
\end{description}

\subsection{Energía Cinética}
En este caso, del gas ideal, la energía cinética de la partícula es toda su energía o Hamiltoniano. Ya sabemos que $K = \flatfrac{p^2}{2m}$
    $$ g(p) \dd{p} = \sqrt{\frac{2}{\pi}} \qty(\frac{\beta}{m})^{\flatfrac{3}{2}} p^2 e^{-\frac{\beta p^2}{2m}} \dd{p}, $$
reemplazando todo por la energía cinética, se tiene $h(K) = $ densidad de probabilidad de Energía Cinética
    $$ h(K) = \frac{2}{\sqrt{\pi}} \beta ^{\flatfrac{3}{2}} \sqrt{K} e^{-\beta K}. $$
\begin{description}
    \item[Moda: ] $K = \frac{\beta}{2}$.
    \item[Valor Esperado: ] $\expval{K} = \expval{E} = \varepsilon = \frac{3}{2\beta} = \frac{3}{2} k_B T = -\dv{\log{\partition}}{\beta} $.
\end{description}

\chapter{Modelo de Einstein}
En un sólido, la energía puede estar almacenada en vibraciones de los átomos (ordenados en forma de \textbf{lattice}). De igual forma que los fotones son ondas electromagnéticas cuantizadas \dsnote{revisar esta palabra que ando bien pendejo y no recuerdo como se dice xd}, los \textbf{fonones} son ondas reticulares cuantifiadas (quantized lattice waves) que describen las exitaciones elementales de vibraciones de la lattice. En lugar de concentrarnos en la vibración de cada átomo, nos concetraremos en los modos normales del sistema. Cada uno de los modos normales es un oscilador armónico simple y contiene un número entero de cuantos de energía. Estos cuantos de energía pueden ser considerados \" partículas \" discretas, llamadas \textbf{fonones}. Para estas descripciones se tiene dos modelos: el de Einstein y el de Debye.


\section{Calor Específico de un Sólido}
En un sólido, las oscilaciones de un átomo son pequeñas y siguen las reglas de la Mecánica Cuántica $E_n = \hbar \omega (n + 1/2).$ (\dsnote{Hasta después se verá bien la mecánica cuántica}.) \\

Calculando el calor específico, utilizando la función de partición 
    $$ \partition (\beta) = \sum _{n_x = 0} ^\infty \sum _{n_y = 0} ^\infty \sum _{n_z = 0} ^\infty e^{-\beta (E_{n_x} + E_{n_y} + E_{n_z})} $$
debido a que $n_x. n_y, n_z$ son índices mudos tenemos que
    $$ \partition (\beta) = \qty(\sum _{n=0} ^\infty e^{-\hbar \omega (n + 1/2)})^3  $$
operando igual que el ejemplo tenemos que
    $$ \partition (\beta) = \qty(\frac{e^{-\frac{\beta \hbar \omega}{2}}}{1 - e^{-\beta \hbar \omega}})^3 . $$
Calculando el logaritmo, su derivada, desarrollando un poco y tomando temperaturas altas se concluye que $ c_v = 3k_B$.

\chapter{Gas Diatómico}
Luego de trabajar con gases ideales monoatómicos y sólidos, se trabajará con el gas diatómico, encontrando el calor específico y la capacidad calorífica. Recordando que para el gas ideal la funciónd de partición
	$$ \partition (\beta) = \frac{V}{\hbar ^3} \qty(\frac{m}{2\beta \pi})^{\flatfrac{3}{2}}, $$
	$$ \log{\partition} = \log{V} - 3\log{\hbar} + \frac{3}{2}\log{\frac{m}{2\pi} - \frac{3}{2} \log{\beta}}. $$

con esto, para un gas ideal el calor específico es $c_v = \frac{3}{2} k_B$ y la capacidad calorífica molar es $C_v = N_A c_v$.

\section{Cálculo de la Función de Partición para un Gas Diatómico}
La molécula diatómicatiene tres tipos de energía
\begin{enumerate}
	\item Energía cinética de traslación
	\item Energía cinética de rotación
	\item Energía cinética de vibración
\end{enumerate}

El índice elegido será: $i = (n_x,n_y,n_z,l,m,n)$; por lo que la función de partición queda de la siguiente forma
	$$ \partition (\beta) = \partition _T \partition _R \partition _V, $$
la que incluye las funciones por traslación, rotación y vibración. La única conocida es la función de partición para la traslación \dsnote{Misma que el gas ideal}.

\subsection{Cálculo de $\partition _R (\beta)$}
El índice es $(l,m)$
	$$ \partition _R (\beta) = \sum _{l=0} ^\infty \sum _{m= -l} ^l e^{-\beta E_{l,m}}, $$
paro $L^2 \ket{l,m} = \hbar ^2 l(l + 1)$ (\dsnote{Esto es de cuántica, se entenderá bien en la siguiente parte.}) La energía cinética de rotación $\frac{1}{2} I \omega$ pero $L = I \omega$ por lo tanto $E_R = \frac{L^2}{2I}$
	$$ E_{l,m} = \frac{\hbar ^2}{2I} l(l+1) $$
o sea que
	$$ \partition _R (\beta) = \sum _{l=0} ^\infty \sum _{m= -l} ^l \exp{-\frac{\beta \hbar ^2}{2I} l(l+1)}. $$
tenemos $2l+1$ $m$'s, entonces
	$$ \partition _R = \sum _{l=0} ^\infty (2l + 1) \exp{-\frac{\beta \hbar ^2}{2I} l(l+1)}, $$
integrando (se tiene el diferencial completo) se tiene
	$$ \partition _R = \frac{2I k_B T}{\hbar ^2}, $$
	$$ \dv[2]{\log{\partition _R}}{\beta} = \frac{1}{\beta ^2} = k_B ^2 T^2. $$


\subsection{Cálculo de $\partition _V (\beta)$}
En este caso no hay degeneración
	$$ \partition _V (\beta) = \sum _{n=0} ^\infty e^{-\beta E_n}, $$
pero $E_n = \hbar \omega (n + 1/2)$. Reemplazando y simplificando
	$$ \partition _V (\beta) = \frac{e^{-\frac{\beta \hbar \omega}{2}}}{1 - e^{-\beta \hbar \omega}} = \frac{1}{\senh{\frac{\beta \hbar \omega}{2}}}, $$
simplificando para $T \to \infty$
	$$ \partition _V (\beta) \approx \frac{1}{\hbar \omega \beta}, $$
	$$ \dv{\log{\partition _V}}{\beta} = -\frac{1}{\beta}. $$


\subsection{Cálculo para $T$ grande}
Se tiene
	$$ \dv{\log{\partition}}{\beta} = -\frac{3}{2\beta} - \frac{1}{\beta} - \frac{1}{\beta} = \frac{7}{2\beta}, $$
	$$ \dv[2]{\log{\partition}}{\beta} = \frac{7}{2\beta ^2} = \frac{7k_B ^2 T^2}{2}. $$
Por ende $c_v = \frac{7k_B}{2}$. La capacidad calorífica molar es $C_V = N_A c_v$.

\subsection{Capacidad Calorífica para Gases Nobles}
Para gases nobles como el helio, solamente se toman en cuenta la traslación y rotación.
	$$ \partition (\beta) = \partition _T \partition _R $$
entonces $c_v = \frac{5}{2} k_B$.
	
\chapter{Repaso de Termodinámica}

\section{Generalidades}
Se repasarán conceptos termodinámicos a partir de la función de partición. La función de partición utilizada hasta ahora ha sido $\partition$ la cual depende de $\beta$ y ahora también consideraremos que depende del volumen. Ahora el volumen también será variable. Si por medio de un baño térmico logrmos mantener la temperatura constante y a flujo de energía cambia el volumen o la presión. \\

Recordemos que para definir la entropía $S$ la dividimos por el número de partículas $n$. Para tener entropía total, tenemos la siguiente variable
	$$ S_B = nS \qquad \text{y} \qquad E = n\varepsilon . $$
Recordemos que
	$$ S = k_B \qty(\beta \varepsilon + \log{\partition}) \quad \Rightarrow \quad S_B = k_B \qty(\beta E + \log{\partition ^n}), $$
y definimos otra variable, que es la función de partición $\mathcal{Z} = \partition ^n$, por ende
	$$ \boxed{ S_B = \qty(\beta E + \log{\bigpartition}) } $$
Factorizando
	$$ S_B = k_B \beta \qty(E + \frac{\log{\bigpartition}}{\beta}), $$
y definimos la Energía Libre de Helmholtz de la siguiente forma
	$$ \boxed{ A = -\frac{\log{\bigpartition}}{\beta} . } $$
y observemos que $A = -n\frac{\log{\partition}}{\beta}$ pero $F = -\frac{\log{\partition}}{\beta}$ donde $F$ es la función de Helmholtz que es un promedio de energía libre. Por lo tanto
	$$ \boxed{A = nF} $$
Reemplazando en la entropía
	$$ \boxed{E = TS_B + A.} $$
Observemos lo siguiente
	$$ S_B = k_B \qty(\beta E + \log{\bigpartition}) $$
ahora consideramos que $\bigpartition (\beta ,V)$, encontrando el diferencial
	$$ \dd{S_B} = k_B \qty(\beta \dd{E} + \qty(\pdv{\log{\bigpartition}}{V}) \dd{V}) $$
La entropía $S_B$  no depende de $\beta$, por otro lado
	$$ \dd{S_B} = \qty(\pdv{S_B}{E})_V \dd{E} + \qty(\pdv{S_B}{V}) _E \dd{V} $$
por lo tanto
	$$ \boxed{ T\dd{S_B}  \dd{E} - \qty(\pdv{A}{V})_V \dd{V}. } $$


\section{Presión}
Por definición clásica de presión $F\dd{x} = P(\text{area}) \dd{x} = \dd{W} = -\dd{E}$. Para definir la presión, tenemos que la entropía es constante.
	$$ P = -\qty(\dv{E}{V}) _{S_B} = -\qty(\pdv{E}{V})_{S_B} $$
	$$ P = -\qty(\pdv{E}{V})_{S_B}. $$
Utilizando la energía libre de Helmholtz tenemos que
	$$ P = -\qty(\pdv{A}{V})_T, $$
entonces
	$$ \dd{E} = T\dd{S_B} - P\dd{V}. $$
De la expresión anterior tenemos que la energía total $E$ depende de la entrpía total $E$ y del volumen $V$. $E$ es idenpendiente de la temperatura $T$. También podemos decir que la entropía total $S_B$ depende de la energía total $E$ y del volumen $V$ pero no depende de la temperatura $T$. \\

Ahora trabajamos con la energía libre de Helmholtz y simplificando su diferencial, se tiene
	$$ \dd{A} = -p\dd{V} - S_b \dd{T}, $$
	$$ \dd{A} = \qty(\pdv{A}{V})_T \dd{V} + \qty(\pdv{A}{T})_V \dd{T}. $$
por lo tanto
	$$ \qty(\pdv{A}{V})_T = -P,\qquad \qquad \qty(\pdv{A}{T})_V = -S_B. $$
	
\section{Entalpía}
Introducimos el concepto de entalpía
	$$ H = E + PV $$
Encontrando su diferencial, esta depende de la entropía y la presión
	$$ \dd{H} = T\dd{S_B} + V\dd{P}, $$
	$$ \dd{H} = \qty(\pdv{H}{S_B})_P \dd{S_B} + \qty(\pdv{H}{P})_{S_B} \dd{P}. $$
entonces
	$$ \qty(\pdv{H}{S_B})_P = T, \qquad \qty(\pdv{H}{P})_{S_B} = V. $$


\section{Función de Gibbs}
La función de Gibbs tiene las siguientes formas equivalentes de escribirse
	$$ G = E + PV + TS_B, $$
	$$ G = H - TS_B, $$
	$$ G = A + PV. $$
Encontrando su diferencial
	$$ \dd{G} = V\dd{P} - S_B \dd{T}, $$
	$$ \dd{G} = \qty(\pdv{G}{P})_T \dd{P} + \qty(\pdv{G}{T})_P \dd{T}. $$
entonces
	$$ \qty(\pdv{G}{P})_T = V, $$
	$$ \qty(\pdv{G}{T})_P = -S_B . $$


\section{Potencial Químico}
Si se agrega una partícula a un sistema, entonces su energía interna cambiará una cantidad que definimos como el \textbf{potencial químico} $\mu$. Así que cuando este es el caso la primera y segunda ley de la termodinámica se deben modificar, agregando un término extra: \dsnote{aguas, me valió el cambio de notación xdddd}
	$$ \dd{U} = T\dd{S_B} - P\dd{V} + \mu \dd{N}, $$
donde $N$ es el número de partículas del sistema. Esto inmediatamente implica que podemos escribir
	$$ \mu = \qty(\pdv{U}{N})_{S,V} . $$
Recordemos que la energía libre de Helmholtz se define como $A = U + TS_B$ y la energía libre de Gibbs como $G = U - PV - TS_B$, por consiguiente
	$$ \dd{F} = -P\dd{V} - S_B \dd{T} + \mu \dd{N} $$
	$$ \dd{G} = V\dd{P} - S_B \dd{T} + \mu\dd{N}, $$
con lo cual se tiene
	$$ \mu = \qty(\pdv{F}{N})_{V,T} $$
	$$ \mu = \qty(\pdv{G}{N})_{P,T}, $$
de manera que esta útlima expresión para $\mu$ en términos de la energía libre de Gibbs se volverá particularmente útil dado que mantener las variables $P$ y $T$ constantes es algo viable en el experimento.\\

Podemos considerar que la función de entropía es $S_B = S_B (U,V,N)$, de tal forma que
	$$ \dd{S_B} = \qty(\pdv{S_B}{U})_{N,V} \dd{U} + \qty(\pdv{S_B}{V})_{N,U} \dd{V} + \qty(\frac{S_B}{N})_{U,V} \dd{N}. $$
Si dividimos la ecuación de la primera ley dentro de $T$ y despejamos para $\dd{S_B}$ se tiene
	$$ \dd{S_B} = \frac{\dd{U}}{T} + \frac{P\dd{V}}{T} - \frac{\mu \dd{N}}{T}, $$
y al compararlo con la ecuación anterior podemos concluir que
	$$ \qty(\pdv{S_B}{U})_{N,V} = \frac{1}{T} \quad \qquad \qty(\pdv{S_B}{V})_{N,U} = \frac{P}{T} \quad \qquad \qty(\pdv{S_B}{N})_{U,V} = -\frac{\mu}{T}. $$
	
	


\chapter{Distribuciones de Fermi-Dirac y Bose-Einstein}
Hasta ahora hemos trabajado con partículas iguales pero distinguibles. Ahora vamos a considerar partículas iguales (idénticas) que son indistinguibles, por ejemplo; los electrones, los neutrinos, los fotones, etc. Tenemos dos tipos de partículas para el caso de partículas indistinguibles, estos dos tipos son los Fermiones y Bosones.

\begin{description}
	\item[Fermiones: ] Los fermiones son partículas indistinguibles que en un estado dado no puede haber más de dos partículas; es decir, en un estado dado o está desocupado o solamente hay una partícula.
	\item[Bosones: ] Los bosones son partículas que en un estado dado puede haber cualquier número de partículas.
\end{description}

En esto se tendrá como índice a la energía.

\section{Degeneración de la Energía}
\begin{tcolorbox}
	$g_i = $ Número de Estados con energía $E_i$.
\end{tcolorbox}


\section{Distribución Fermi-Dirac (Caso 3)}
$n_i = $ Número de partículas en el estado $i$. En este caso vamos a considerar Fermiones. Debido a esto tenemos $0 \leq n_i \leq g_i$. ¿De cuantas formas se pueden colocar $n_i$ partículas en el nivel de energía $E_i$ que tiene $g_i$ estados? \\

Tenemos que elegir $n_i$ estados de $g_i$ estados disponibles; donde no se puede y no importa el orden (caso 3). Por lo tanto el resultado es
	$$ \prescript{}{g_i}{C}_{n_i} = \mqty(g_i \\ n_i) = \frac{g_i !}{n_i ! (g_i - n_i)!}. $$
$\Omega = $ Número de microestados.
	$$ \Omega = \prod _{i=1} ^k \mqty(g_i \\ n_i) = \prod _{i=1} ^k \frac{g_i !}{n_i ! (g_i - n_i)!}. $$
Queremos maximizar $\Omega$ sujeta a dos restricciones $\sum n_i = n$ y $\sum n_i E_i = E_T$. Además, si maximizamos $\log{\Omega}$, tambien maximizamos $\Omega$. Vamos a calcular $\frac{n_i}{g_i}$ en lugar de $n_i$. Para resolver el problema anterior utilizaremos la fórmula de Stirling, reemplazandola y aplicando el logaritmo e ignorando $\frac{1}{2} \log{2\pi}$ y los $1/2$, por lo tanto, hay que maximizar
	$$ \log{\Omega} = \sum _{i=1} ^n \qty[g_i \log{g_i} - n_i \log{n_i} - (g_i - n_i)\log{(g_i - n_i)}] $$
aplicamos multiplicadores de Lagrange
	$$ F(n_1,n_2,\ldots,n_k,\alpha,\beta) = \sum _{i=1} ^n \qty[g_i \log{g_i} - n_i \log{n_i} - (g_i - n_i)\log{(g_i - n_i)}] - \alpha \qty(\sum _{i=1} ^k n_i - n) - \beta (\sum _{i=1} ^k n_i E_i - E_T). $$
Ahora $\pdv{F}{n_i} = 0$
	$$ \pdv{F}{n_i} = -\log{n_i} + \log{(g_i - n_i)} - \alpha - \beta E_T. $$
	$$ \log{\qty(\frac{g_i - n_i}{n_i})} = \alpha + \beta E_i $$
introducimos la variable $\mu = -\alpha / \beta$
	$$ \frac{g_i}{n_i} - 1 = e^{\alpha + \beta E_i} $$
seguimos operando
	$$ \frac{n_i}{g_i} = \frac{1}{e^{\beta (E_i - \mu)} + 1} \qquad \text{Fermi-Dirac} $$
entonces
	$$ \sum _{i=1} ^k \frac{g_i E_i}{e^{\beta (E_i - \mu)} + 1} = E_T. $$


\section{Distribución de Bose-Einstein (Caso 4)}
$n_i$ es el número de partículas en el nivel de energía $E_i$. En este caso $n_i \geq 0$, porque son bosones y en un mismo estado puede haber cualquier número de partículas. Osea $n_i$ puede ser mayor que $g_i$. \\

Tenemos que elegir $n_i$ veces $g_i$ estados donde no importa el orden y sí se puede repetir (distribuciones). Tenemos al siguiente número de posibilidades
	$$ \mqty(n_i + g_i - 1 \\ n_i) = \mqty(n_i + g_i - 1 \\ g_i - 1) $$
por lo que
	$$ \mqty(n_i + g_i - 1 \\ n_i) = \frac{(n_i + g_i - 1)}{n_i ! (g_i - 1)!}. $$
Para calcular el número de microestados hacemo uso de la regla del producto y tenemos que
	$$ \Omega = \prod _{i=1} ^k \frac{(n_i + g_i - 1)}{n_i ! (g_i - 1)!} $$
y las mismas condiciones que el caso anterior. Ahora maximizamos $\log{\Omega}$ bajo el mismo procedimiento: fórmula de stirling, selección de términos irrelevantes y multiplicadores de Lagrange.
	$$ \log{\qty(\frac{g_i + n_i - 1}{n_i})} = \alpha + \beta E_i $$
vamos a ignorar el $-1$ y $\mu = \alpha / \beta$
	$$ \frac{n_i}{g_i} = \frac{1}{e^{\beta (E_i - \mu)} - 1} \qquad \text{Bose-Einstein} $$
entonces
	$$ \sum _{i=1} ^k \frac{g_i E_i}{e^{\beta (E_i - \mu)} - 1} = E_T. $$


\chapter{Gas de Fotones}

\section{Radiación de Cuerpo Negro}
Ahora se estudiará la distribución de enegía en una cavidad de volumen $V$ en ondas electromagnéticas. En la cavidad de volumen $V$ la energía está en las ondas electromagnéticas. Una onda electromagnética es el resultado de las oscilaciones del campo eléctrico y el cambpo magnético. Las oscilaciones del campo eléctrico y el magnético obedecen las leyesdde la electrodinámica y de la mecánica cuántica. En partícular las ondas electromagnéticas se comportan según la Teoría Cuántica de Campos; al oscilar los campos obtenemos partículas llamadas \textbf{fotones} que siguen la estadística de Bose-Einstein; los fotones son bosones. Las ondas electromagnéticas, por ende los fotones, viaja a la velocidad de la luz
	$$ c = \frac{1}{\sqrt{\mu _o \varepsilon _o}} = 3\times 10^8 m/s $$
También sabemos que las ondas electromagnéticas son ondas transversales; el campo eléctrico y magnético oscilan perpendicularmente a la dirección de propagación, permitiendo que tanto el campo eléctrico como el magnético tengan dos direcciones linealmente independientes para oscilar; por lo tanto, tenemos dos polarizaciones. \\
Las ondas que se forman en la cavidad de volumen $V$ son ondas estacionarias. Las ondas estacionarias se forman con la superposición de dos ondas simples que viajan en sentidos opuestos. La energía está directamente relacionada con la frecuencia.

\subsection{Cálculo de la Función de Degeneración $g(E)$}
Las ondas estacionarias son la combinación lineal de los modos normales, en general se tiene que $k_n = \frac{3\pi}{L}$; además, sabemos que $\lambda \nu = \frac{\omega}{k} = c$. Si consideramos la propagación en 3-D $k_n = \sqrt{n_x ^2 + n_y ^2 + n_z ^2}$. El índice discreto $n$ está altamente degenerado y vamos a cambiarlo por $n\to \omega _n \to \omega$ para calcular la energía $E_n$ usamos Mecánica Cuántica \dsnote{No se hará el desarrollo, ver Clase 15 Notas de Clase 2022}. Con esto se llega a que la función de degeneración es
	$$ \boxed{ g(\omega) = \frac{V \omega ^2}{\pi ^2 c^3}, } $$
con $g(\omega) \dd{\omega} = $ número de estados entree $\omega$ y $\omega + \dd{\omega}$. Reemplazando $E(\omega) = \hbar \omega$, entonces
	$$ f(E) = \frac{VE^2}{\pi ^2 c^2 \hbar ^3}, $$
con $f(E) \dd{E}$ es el número de estados entre $E$ y $E + \dd{E}$. \\

Debido a que los fotones son bosones, se distribuyen según la estadística de Bose-Einstein, dicho de otra forma, tienen la distribución de Bose-Einstein
	$$ E_T = \sum _{i=1} ^k \frac{g_i E_i}{e^{\beta (E_i - \mu)} - 1} $$
ahora $\alpha = 0$ y si $i \to \omega$ se tiene que $g_i \to g(\omega)$ y $E_i = E(\omega) = \hbar \omega$ entonces
	$$ E_T = \int _0 ^\infty \frac{g(\omega( E(\omega)}{e^{\beta E(\omega)} - 1} \dd{\omega} $$
	$$ \boxed{ E_T = \int _0 ^\infty \frac{V \omega ^2 \hbar \omega \dd{\omega}}{\pi ^2 c^3 \qty(e^{\beta \hbar \omega} - 1)}. } $$
	
Integrando se tiene que
	$$ \boxed{ E_T = \qty(\frac{V\pi^2 k_B ^4}{15 \hbar ^3 c^3}) T^4. } $$


\subsection{Constante de Stefan-Boltzmann}
$\sigma = $ constante de Stefan-Boltzmann
	$$ \sigma = \frac{\pi ^2 k_B ^4}{60 \hbar ^3 c^2}. $$
por lo que la energía total se reescribe como $E_T = \frac{4\sigma V}{c} T^4$.

\subsection{Densidad de Energía}
$e = \frac{E_T}{V} = $ densidad de energía.Regresando a la integral
	$$ e = \frac{\hbar}{\pi^2 c^3} \int _0 ^\infty \frac{\omega ^3 \dd{\omega}}{e^{\beta \hbar \omega} - 1} = \int _0 ^\infty \mu (\omega) \dd{\omega}. $$
	
con esto definimos $\mu (\omega)$
	$$ \mu (\omega) = \frac{\hbar \omega ^3 }{\pi ^2 c^3 (e^{\beta \hbar \omega} - 1)} $$
donde $\mu$ es la energía por unidad de volumen. Y en términos de la frecuencia
	$$ \mu (\nu) = \frac{8\pi h \nu ^3}{c^3 \qty(e^{\frac{h\nu}{k_B T}} - 1)}. $$


\subsection{Conteo de Fotones}
De lo que se estudió en la distribución de Bose-Einstein el número de fotones es
	$$ N = \frac{V}{\pi ^2 c^3} \int _0 ^\infty \frac{\omega ^2 \dd{\omega}}{\qty(e^{\beta \hbar \omega} - 1)}. $$



\subsection{Función de Partición}

Teniendo
	$$\pdv{\log{\bigpartition}}{\beta} = - E_T = - \frac{V \pi ^2}{15 \hbar ^3 c^3} \frac{1}{\beta ^4} $$
Entonces, la función de partición es
	$$ \boxed{ \bigpartition (\beta) = e^{\frac{V \pi ^2}{45 \hbar ^3 c^3 \beta ^3}} .} $$



































%%%%%%%%%%%%%%
